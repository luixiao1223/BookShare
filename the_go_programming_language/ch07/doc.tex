\hypertarget{interface}{%
\subsection{interface}\label{interface}}

Go语言的接口的独特之处在于是\texttt{隐式实现}的。对于一个具体的类型,无须申明实现了哪些接口,只需要提供接口所必须的的方法即可。

\begin{verbatim}
package io

type Reader interface{
    Read(p []byte)(n int, err error)
}

type Closer interface{
    Close() error
}
\end{verbatim}

\hypertarget{interfaceux5d4cux5165}{%
\subsubsection{interface嵌入}\label{interfaceux5d4cux5165}}

\begin{verbatim}
type ReadCloser interface{
    Reader
    Closer
}
\end{verbatim}

\hypertarget{ux6307ux9488ux7c7bux578bux548cux975eux6307ux9488ux7c7bux578bux6240ux542bux6709ux7684ux63a5ux53e3ux4e0dux4e00ux5b9aux540cux76f8}{%
\subsubsection{指针类型和非指针类型所含有的接口不一定同相}\label{ux6307ux9488ux7c7bux578bux548cux975eux6307ux9488ux7c7bux578bux6240ux542bux6709ux7684ux63a5ux53e3ux4e0dux4e00ux5b9aux540cux76f8}}

\begin{verbatim}
 type IntSet struct { /* ... */ }
 func (*IntSet) String() string
 var _ = IntSet{}.String() // compile error: String requires *IntSet receiver
  
 var s IntSet
 var _ = s.String() // OK: s is a variable and &s has a String method
 
 var _ fmt.Stringer = &s // OK
 var _ fmt.Stringer = s  // compile error: IntSet lacks String method
\end{verbatim}

\hypertarget{interfaceux5b9eux73b0}{%
\subsubsection{interface实现}\label{interfaceux5b9eux73b0}}

如果一个具体的类型要实现一个接口,就必须实现接口类型中定义的所有方法。

\hypertarget{ux63a5ux53e3ux7684ux8d4bux503cux89c4ux5219}{%
\paragraph{接口的赋值规则}\label{ux63a5ux53e3ux7684ux8d4bux503cux89c4ux5219}}

仅当一个表达式实现了一个接口时,这个表达式才可以赋值给该接口:

\begin{verbatim}
var M io.Write 
w = os.Stdout           // OK : *os.File 有Write 方法
w = new(bytes.Buffer)   // OK: *bytes.Buffer有Write方法
w = time.Second         //  编译错误: time.Duration缺少Write 方法
\end{verbatim}

\hypertarget{ux7a7aux63a5ux53e3ux7c7bux578b-interface}{%
\paragraph{空接口类型
interface}\label{ux7a7aux63a5ux53e3ux7c7bux578b-interface}}

空接口类型对其实现类型没有任何要求,所以我们可以把任何值赋给空接口类型。

\begin{verbatim}
var any interface{}
any = true
any = 12.34
any = "hello"
any = map[string]int{"one": 1}
any = new(bytes.Buffer)
\end{verbatim}

\hypertarget{ux4f7fux7528flag.value-ux6765ux89e3ux6790ux53c2ux6570}{%
\subsubsection{使用flag.Value
来解析参数}\label{ux4f7fux7528flag.value-ux6765ux89e3ux6790ux53c2ux6570}}

如下一个程序,它实现了睡眠指定时间的功能。通过
-period命令行标志控制睡眠时长。

\begin{verbatim}
var period = flag.Duration("period",l*time.Second,"sleep period")
func main() {
    flag.Parse()
    fmt.Printf ("Sleeping for %v ...",*period)
    time.Sleep(*period)
    fmt.Println()
}

$ ./sleep -period 50ms
Sleeping for 50ms ...
\end{verbatim}

对于时间长度的也可以支持自定义类型,只需要满足flag.Value接口的类型。

\begin{verbatim}
package flag
//Value 接口代表了存储在标志内的值
type Value interface {
    String() string
    Set(string) error
}
\end{verbatim}

\hypertarget{ux63a5ux53e3ux503c}{%
\subsubsection{接口值}\label{ux63a5ux53e3ux503c}}

接口类型的值(接口值),分为两个部分:具体的类型和该类型的值。在Go语言中用类型描述符来表述接口值的类型部分。

\begin{verbatim}
var w io.Write  
w = os.Stdout   
w = new(bytes.Buffer)
w = nil
\end{verbatim}

在上述的四行代码中,变量w有三不同的值。
第一行是变量的初始化,这里将接口的类型和值都设置为nil
第二行把\emph{os.File类型赋值给w,w的接口值的动态类型变成了
}os.File的类型描述符,动态值设置为了一个指向代表进程的标准输出的os.File
类型的指针。
第三行把\emph{bytes.Buffer类型赋值给w,w的接口值的动态类型变成了
}bytes.Buffer类型描述符,动态值变成了一个指向新分配缓冲区的指针。
第四行又把nil赋值给了w。

\emph{接口值可以用==和!=操作符来做比较,在比较两个接口值时,如果两个接口值的动态类型一致,但对应的动态值是不可比较的(比如slice
),那么这个比较会以崩溃的方式失败}

可以使用fmt包的\%T来拿到接口值的的动态类型,这在处理错误和调试时很有帮助:

\begin{verbatim}
w = os.Stdout
fmt.Printf("%T\n", w) //"*os.File"
\end{verbatim}

\hypertarget{ux6ce8ux610fux542bux6709ux7a7aux6307ux9488ux7684ux975eux7a7aux63a5ux53e3}{%
\paragraph{注意:含有空指针的非空接口}\label{ux6ce8ux610fux542bux6709ux7a7aux6307ux9488ux7684ux975eux7a7aux63a5ux53e3}}

\textbf{空的接口值(其中不包含任何信息)与仅仅动态值为nil
的接口值是不一样的.}

\begin{verbatim}
const debug = true
func main() {
    var buf *bytes.Buffer  //var buf io.Writer 这样定义没有错
    if debug {
        buf = new(bytes.Buffer) //启用输出收集
    }
    
    f(buf) //注意: 微妙的错误 当debug为false时,buf是的动态值为nil
    if debug {
        //...使用buf...
    }
}
//如果out 不是nil , 那么会向其写入输出的数据
func f(out io.Writer) {
//...其他代码...
    if out != nil {
    out.Write([]byte("done!\n") //向一个空接收者写值,崩溃。
    }
}
\end{verbatim}

\hypertarget{ux4f7fux7528sort.interface-ux6765ux6392ux5e8f}{%
\subsubsection{使用sort.Interface
来排序}\label{ux4f7fux7528sort.interface-ux6765ux6392ux5e8f}}

sort 包提供了针对任意序列根据任意排序函数原地排序的功能。

\begin{verbatim}
package sort
type Interface interface { 
Len() int 
Less(i, j int) bool // i, 是序列元素的
Swap(i, j int)
}
\end{verbatim}

\hypertarget{http.-handler-ux63a5ux53e3}{%
\subsubsection{http. Handler 接口}\label{http.-handler-ux63a5ux53e3}}

\begin{verbatim}
package http
type Handler interface {
    ServeHTTP(w ResponseWriter, r *Request)
}
func ListenAndServe(address string, h Handler) error
\end{verbatim}

ListenAndServe
函数需要一个服务器地址,比如``localhost:8000'',以及一个Handler接口的实例(用来接受所有的请求)。

示例:给定一个服务器地址,输入DB中的信息

\begin{verbatim}
func main() {
    db := database {"shoes ":50 ," socks":5}
    log.Fatal(http.ListenAndServe("localhost:8000",db))
}
type dollars Float32
func (d dollars) String() string {return fmt.Sprintf (”$% . 2f”, d) }
type database map[string]dollars
func (db database) ServeHTTP(w http.ResponseWriter,req *http.Request) { 
    for item, price := range db {
        fmt.Fprintf(w, "%s: %s\n", item, price)
    }
}
req..URL.Path
req.URL.Query.get("")
······
\end{verbatim}

\#\#\# error接口

\begin{verbatim}
type Error interface {
   Error() string
}
\end{verbatim}

构造error最简单的方法就是调用errors.New

完整的error包只有4行代码:

\begin{verbatim}
package errors
func New(text string) error{ return &errorString{text} }
type errorString struct { text string }
func (e *errorString) Error() string { return e.text }
\end{verbatim}

通常使用封装函数fmt.Errorf来构造error。

\hypertarget{ux7c7bux578bux65adux8a00}{%
\subsubsection{类型断言}\label{ux7c7bux578bux65adux8a00}}

形式:x.(T) //x是一个接口类型的表达式,T是一个类型

类型断言会检查作为操作数的动态类型是否满足指定的断言类型。

\begin{verbatim}
var w io.Writer
w = os.Stdout
f := w.(*os.File)  //成功: f = = os.Stdout
c : = w.(*bytes.Buffer) // 崩溃:接口持有的是*os. File,不是*bytes.Buffer
\end{verbatim}

如果断言出现在需要两个结果的赋值表达式中,那么断言不会在失败时崩溃:

\begin{verbatim}
var w io.Writer = os.Stdout // luixiao1223: os.Stdout是一个值。 io.Writer是类型。
f , ok : = w.(*os.File)    //成功:ok, f == os.Stdout
b, ok := w.(*bytes.Buffer)// 失败:!ok, b == nil
\end{verbatim}
