\hypertarget{ux4f7fux7528ux5171ux4eabux53d8ux91cfux5b9eux73b0ux5e76ux53d1}{%
\section{9.
使用共享变量实现并发}\label{ux4f7fux7528ux5171ux4eabux53d8ux91cfux5b9eux73b0ux5e76ux53d1}}

\hypertarget{ux7adeux6001}{%
\subsection{9.1 竞态}\label{ux7adeux6001}}

\hypertarget{ux5e76ux53d1ux4e00ux4e2aux7a0bux5e8fux53eaux6709ux4e00ux4e2a-goroutine-ux65f6ux7a0bux5e8fux4e32ux884cux8fdbux884cux6709ux591aux4e2a-goroutine-ux65f6ux91ccux9762ux7684ux4e8bux4ef6ux5e76ux53d1ux8fdbux884c}{%
\subsubsection{并发:一个程序只有一个 goroutine 时,程序串行进行;有多个
goroutine
时,里面的事件并发进行}\label{ux5e76ux53d1ux4e00ux4e2aux7a0bux5e8fux53eaux6709ux4e00ux4e2a-goroutine-ux65f6ux7a0bux5e8fux4e32ux884cux8fdbux884cux6709ux591aux4e2a-goroutine-ux65f6ux91ccux9762ux7684ux4e8bux4ef6ux5e76ux53d1ux8fdbux884c}}

\hypertarget{ux5e76ux53d1ux5b89ux5168ux51fdux6570ux5728ux5e76ux53d1ux8c03ux7528ux65f6ux80fdux6b63ux5e38ux5de5ux4f5c}{%
\subsubsection{并发安全:函数在并发调用时能正常工作}\label{ux5e76ux53d1ux5b89ux5168ux51fdux6570ux5728ux5e76ux53d1ux8c03ux7528ux65f6ux80fdux6b63ux5e38ux5de5ux4f5c}}

\hypertarget{ux7adeux6001ux591aux4e2a-goroutine-ux6309ux67d0ux4e9bux4ea4ux9519ux987aux5e8fux6267ux884cux65f6ux7a0bux5e8fux65e0ux6cd5ux7ed9ux51faux6b63ux786eux7ed3ux679c}{%
\subsubsection{竞态:多个 goroutine
按某些交错顺序执行时,程序无法给出正确结果}\label{ux7adeux6001ux591aux4e2a-goroutine-ux6309ux67d0ux4e9bux4ea4ux9519ux987aux5e8fux6267ux884cux65f6ux7a0bux5e8fux65e0ux6cd5ux7ed9ux51faux6b63ux786eux7ed3ux679c}}

\hypertarget{ux4e3eux4f8bux94f6ux884cux8d26ux6237}{%
\subsubsection{举例:银行账户}\label{ux4e3eux4f8bux94f6ux884cux8d26ux6237}}

\begin{verbatim}
package bank
var balance int
func Deposit(amount int) { balance = balance + amount }
func Balance() int {return balance }

// Alice:
go func () {
    bank.Deposit(200) // Al
    fmt.Println("=", bank. Balance()) // A2
}()
// Bob :
go bank.Deposit(100) // B
\end{verbatim}

\begin{itemize}
\tightlist
\item
  程序中的这种状况是竞态中的一种,称为数据竞态(data
  race)。数据竞态发生于两个 goroutine
  并发读写同一个变量并且至少其中一个是写入时
\item
  Go 语言很少有``未定义行为''的问题
\end{itemize}

\begin{verbatim}
var x []int
go func() { x = make([]int, 10) }()
go func() { x = make([]int, 1000000) }()
x[999999) = 1 //注意:未定义行为,可能造成内存异常
\end{verbatim}

\hypertarget{ux907fux514dux6570ux636eux7adeux6001ux7684ux4e09ux79cdux65b9ux6cd5}{%
\subsubsection{避免数据竞态的三种方法}\label{ux907fux514dux6570ux636eux7adeux6001ux7684ux4e09ux79cdux65b9ux6cd5}}

\begin{itemize}
\tightlist
\item
  不要修改变量
\end{itemize}

\begin{verbatim}
var icons = make(map[string]image.Image)
func loadlcon(name string) image.Image
// 注意:并发不安全
func Icon(name string) image.Image {
    icon, ok := icons[name]
    if !ok {
        icon = loadIcon(name)
        icons[name] = icon
    }
    return icon
}

var icons = map[string]image.Image{
    "spades.png": load Icon("spades.png”),
    "hearts.png":load Icon ("hearts.png"),
    "diamonds.png": load Icon("diamonds.png"),
    "clubs.png" : loadlcon("clubs.png”),
}
// 并发安全
func Icon(name string) image.Image { return icons[name] }
\end{verbatim}

\begin{itemize}
\tightlist
\item
  避免从多个 goroutine 访问同一个变量
\end{itemize}

\begin{verbatim}
// Package bank provides a concurrency-safe bank with one account.
package bank

var deposits = make(chan int) // send amount to deposit
var balances = make(chan int) // receive balance

func Deposit(amount int) { deposits <- amount }
func Balance() int       { return <-balances }

func teller() {
    var balance int // balance 被限制在 teller goroutine 中
    for {
        select {
        case amount := <-deposits:
            balance += amount
        case balances <- balance:
        }
    }
}

func init() {
    go teller() // start the monitor goroutine
}
\end{verbatim}

\begin{itemize}
\tightlist
\item
  允许多个 goroutine 访问同一个变量,但在同一时间只有一个goroutine
  可以访问。这种方法称为互斥机制。
\end{itemize}

\hypertarget{ux4e92ux65a5ux9501sync.mutex}{%
\subsection{9.2 互斥锁:sync.Mutex}\label{ux4e92ux65a5ux9501sync.mutex}}

\hypertarget{ux4e00ux4e2aux8ba1ux6570ux4e0aux9650ux4e3a1ux7684ux4fe1ux53f7ux91cfux79f0ux4e3aux4e8cux8fdbux5236ux4fe1ux53f7ux91cf-binary-semaphore}{%
\subsubsection{一个计数上限为1的信号量称为二进制信号量 (binary
semaphore)}\label{ux4e00ux4e2aux8ba1ux6570ux4e0aux9650ux4e3a1ux7684ux4fe1ux53f7ux91cfux79f0ux4e3aux4e8cux8fdbux5236ux4fe1ux53f7ux91cf-binary-semaphore}}

\hypertarget{sync-ux5305ux6709ux4e00ux4e2aux5355ux72ecux7684-mutex-ux7c7bux578bux6765ux652fux6301ux4e92ux65a5ux9501ux6a21ux5f0fux5b83ux7684-lock-ux65b9ux6cd5ux7528ux4e8eux83b7ux53d6ux4ee4ux724ctokenux6b64ux8fc7ux7a0bux4e5fux79f0ux4e3aux4e0aux9501unlock-ux65b9ux6cd5ux7528ux4e8eux91caux653eux4ee4ux724c}{%
\subsubsection{sync 包有一个单独的 Mutex 类型来支持互斥锁模式。它的 Lock
方法用于获取令牌(token,此过程也称为上锁),Unlock
方法用于释放令牌:}\label{sync-ux5305ux6709ux4e00ux4e2aux5355ux72ecux7684-mutex-ux7c7bux578bux6765ux652fux6301ux4e92ux65a5ux9501ux6a21ux5f0fux5b83ux7684-lock-ux65b9ux6cd5ux7528ux4e8eux83b7ux53d6ux4ee4ux724ctokenux6b64ux8fc7ux7a0bux4e5fux79f0ux4e3aux4e0aux9501unlock-ux65b9ux6cd5ux7528ux4e8eux91caux653eux4ee4ux724c}}

\begin{verbatim}
// Package bank provides a concurrency-safe single-account bank.
package bank

//!+
import "sync"

var (
    mu      sync.Mutex // guards balance
    balance int
)

func Deposit(amount int) {
    mu.Lock()
    balance = balance + amount
    mu.Unlock()
}

func Balance() int {
    mu.Lock()
    b := balance
    mu.Unlock()
    return b
}
\end{verbatim}

\begin{itemize}
\tightlist
\item
  在 Lock 和 Unlock
  之间的代码,可以自由地读取和修改共享变量,这一部分称为临界区域
\item
  这种函数、互斥锁、变量的组合方式称为监控(monitor)模式
\item
  \textbf{Go语言的 defer 语句}:通过延迟执行 Unlock
  就可以把临界区域隐式扩展到当前函数的结尾,避免了必须在一个或者多个远离
  Lock 的位置插入一条 Unlock 语句
\end{itemize}

\begin{verbatim}
func Balance() int {
    mu.Lock()
    defer mu.Unlock()
    return balance
}
\end{verbatim}

\begin{itemize}
\tightlist
\item
  另一个问题:下面的代码 Deposit 会通过调用 mu.Lock()
  来尝试再次获取互斥锁,但由于互斥锁是不能再入的(无法对一个已经上锁的五斥量再上锁),因此这会导致死锁,Withdraw
  会一直被卡住
\end{itemize}

\begin{verbatim}
func Withdraw(amount int) bool {
    mu.Lock()
    defer mu.Unlock()
    Deposit(-amount)
    if Balance() < 0 {
        Deposit (amount)
        return false //余额不足
    }
    return true
}
\end{verbatim}

解决方法:

\begin{verbatim}
func Withdraw(amount int) bool {
    mu.Lock()
    defer mu.Unlock()
    deposit(-amount)
    if Balance() < 0 {
        deposit (amount)
        return false //余额不足
    }
    return true
}

func Deposit(amount int) {
    mu.Lock()
    defer mu.Unlock()
    deposit (amount)
}
\end{verbatim}

\hypertarget{ux8bfbux5199ux4e92ux65a5ux9501sync.rwmutex}{%
\subsection{9.3
读写互斥锁:sync.RWMutex}\label{ux8bfbux5199ux4e92ux65a5ux9501sync.rwmutex}}

\begin{verbatim}
var mu sync.RWMutex
var balance int

func Balance() int {
    mu.RLock()
    defer mu.RUnlock()
    return balance
}
\end{verbatim}

\begin{itemize}
\tightlist
\item
  Rlock
  仅可用于在临界区域内对共享变量元写操作的情形。一般来讲,我们不应当假定那些逻辑上只读的函数和方法不会更新一些变量
\item
  因为复杂的内部簿记工作,仅在绝大部分 goroutine
  都在获取读锁并且锁竞争比较激烈时(即,goroutine
  一般都需要等待后才能获到锁),RWMutex 才有优势
\end{itemize}

\hypertarget{ux5185ux5b58ux540cux6b65}{%
\subsection{9.4 内存同步}\label{ux5185ux5b58ux540cux6b65}}

\begin{verbatim}
var x, y int
go func() {
    x = 1 // A1
    fmt.Print ("y:", y, " ") // A2
}()
go func() {
    y = 1 // A1
    fmt.Print ("x:", x, " ") // A2
}()
\end{verbatim}

可能的结果: y:0 x:1 x:0 y:1 x:1 y:1 y:1 x:1 x:0 y:0 y:0 x:0

\begin{itemize}
\tightlist
\item
  goroutine 是串行一致的(sequentially
  consistent),但在缺乏使用通道或者互斥量来显式同步的情况下,并不能保证所有的
  goroutine 看到的事件顺序都是一致的
\end{itemize}

\hypertarget{ux5ef6ux8fdfux521dux59cbux5316sync.once}{%
\subsection{9.5
延迟初始化:sync.Once}\label{ux5ef6ux8fdfux521dux59cbux5316sync.once}}

\begin{verbatim}
var icons map[string]image.Image

func loadIcons() {
    icons = map[string]image.Image{
        "spades.png": loadlcon("spades.png”),
        "hearts.png":load Icon ("hearts.png"),
        "diamonds.png": load Icon("diamonds.png"),
        "clubs.png" : loadlcon("clubs.png”),
    }
}

// 注意:并发不安全
func Icon(name string) image.Image {
    if icons == nil {
        loadIcons()
    }
    return icon[name]
}
\end{verbatim}

语句可能被如下重排:

\begin{verbatim}
func loadIcons() {
    icons = make(map[string]image.Image)
    icons["spades.png"] = load Icon("spades.png”),
    icons["hearts.png"] = load Icon ("hearts.png"),
    icons["diamonds.png"] = load Icon("diamonds.png"),
    icons["clubs.png"] = load Icon("clubs.png”),
}
\end{verbatim}

解决方法:

\begin{verbatim}
// 注意:并发安全
func Icon(name string) image.Image {
    mu.Lock()
    defer mu.Unlock()
    if icons == nil {
        loadIcons()
    }
    return icon[name]
}
\end{verbatim}

使两个 goroutine 可以并发访问 icons:

\begin{verbatim}
// 注意:并发安全
func Icon(name string) image.Image {
    mu.RLock()
    if icons != nil {
        icon := icons[name]
        mu.RUnlock()
        return icon
    }
    mu.RUnlock()
    
    mu.Lock()
    if icons == nil {
        loadIcons()
    }
    icon := icons[name]
    mu.Unlock()
    return icon
}
\end{verbatim}

\begin{itemize}
\tightlist
\item
  sync.Once 提供了这种一次性初始化问题的解决方案
\item
  Once
  包含一个布尔变量和一个互斥量,布尔变量记录初始化是否已经完成,互斥量则负责保护这个布尔变量和客户端的数据结构
\end{itemize}

\begin{verbatim}
var loadIconsOnce sync.Once
var icons map[string]image.Image

// 并发安全
func Icon(name string) image.Image {
    loadiconsOnce.Do(loadicons)
    return icons[name]
}
\end{verbatim}

\subsection{9.6 竞态检测器}

\begin{itemize}
\tightlist
\item
  简单地把 race 命令行参数加到 go build 、go run 、go test
  命令里边即可使用竞态检测器功能
\item
  它会让编译器为你的应用或测试构建一个修改后的版本,它使用额外的手法高效记录在执行时对共享变量的所有访问,以及读写这些变量的goroutine
  标识,以及所有的同步事件
\end{itemize}

\hypertarget{ux793aux4f8b}{%
\subsection{9.7 示例}\label{ux793aux4f8b}}

\hypertarget{goroutine-ux4e0eux7ebfux7a0b}{%
\subsection{9.8 goroutine 与线程}\label{goroutine-ux4e0eux7ebfux7a0b}}

\hypertarget{ux672cux8d28ux4e0aux4e24ux8005ux5deeux522bux662fux91cfux53d8}{%
\subsubsection{本质上两者差别是量变}\label{ux672cux8d28ux4e0aux4e24ux8005ux5deeux522bux662fux91cfux53d8}}

\hypertarget{ux4e09ux4e2aux533aux5206ux70b9}{%
\subsubsection{三个区分点:}\label{ux4e09ux4e2aux533aux5206ux70b9}}

\begin{enumerate}
\def\labelenumi{\arabic{enumi}.}
\tightlist
\item
  可增长的栈
\item
  goroutine 调度:Go 包含自己的一个调度器,由特定 Go 语言结构出发
\item
  GOMAXPROCS:Go 调度器使用 GOMAXPROCS 参数来确定需要使用多少个 OS
  线程来同时执行 Go 代码
\end{enumerate}
