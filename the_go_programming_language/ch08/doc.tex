\hypertarget{goroutine-1}{%
\section{goroutine}\label{goroutine-1}}

\begin{enumerate}
\tightlist
\item
  goroutine和线程之间再数量上有很大的区别 fixme (9.8节)
\item
  main函数其实就是一个goroutine,我们把它称作主goroutine.
\end{enumerate}

\begin{Shaded}
\begin{Highlighting}[]
\NormalTok{f()}
\KeywordTok{go}\NormalTok{ f()}
\end{Highlighting}
\end{Shaded}

\begin{verbatim}
#include <iostream>
// 必须的头文件
#include <pthread.h>

using namespace std;

#define NUM_THREADS 5

// 线程的运行函数
void* say_hello(void* args)
{
    cout << "Hello Runoob!" << endl;
    return 0;
}

int main()
{
    // 定义线程的 id 变量,多个变量使用数组
    pthread_t tids[NUM_THREADS];
    for(int i = 0; i < NUM_THREADS; ++i)
    {
        //参数依次是:创建的线程id,线程参数,调用的函数,传入的函数参数
        int ret = pthread_create(&tids[i], NULL, say_hello, NULL);
        if (ret != 0)
        {
            cout << "pthread_create error: error_code=" << ret << endl;
        }
    }
    //等各个线程退出后,进程才结束,否则进程强制结束了,线程可能还没反应过来;
    pthread_exit(NULL);
}
\end{verbatim}

我们先来看看一个具体的例子

\begin{Shaded}
\begin{Highlighting}[]
 \KeywordTok{func}\NormalTok{ main() \{}
  \KeywordTok{go}\NormalTok{ spinner(}\DecValTok{100}\NormalTok{ * time.Millisecond)}
  \KeywordTok{const}\NormalTok{ n = }\DecValTok{45}
\NormalTok{  fibN := fib(n) }\CommentTok{// slow}
\NormalTok{  fmt.Printf(}\StringTok{"}\CharTok{\textbackslash{}r}\StringTok{Fibonacci(%d) = %d}\CharTok{\textbackslash{}n}\StringTok{"}\NormalTok{, n, fibN)}
\NormalTok{\}}

\KeywordTok{func}\NormalTok{ spinner(delay time.Duration) \{}
  \KeywordTok{for}\NormalTok{ \{}
    \KeywordTok{for}\NormalTok{ _, r := }\KeywordTok{range} \StringTok{`-\textbackslash{}|/`}\NormalTok{ \{}
\NormalTok{      fmt.Printf(}\StringTok{"}\CharTok{\textbackslash{}r}\StringTok{%c"}\NormalTok{, r)}
\NormalTok{      time.Sleep(delay)}
\NormalTok{    \}}
\NormalTok{  \}}
\NormalTok{\}}

\KeywordTok{func}\NormalTok{ fib(x }\DataTypeTok{int}\NormalTok{) }\DataTypeTok{int}\NormalTok{ \{}
  \KeywordTok{if}\NormalTok{ x < }\DecValTok{2}\NormalTok{ \{}
    \KeywordTok{return}\NormalTok{ x}
\NormalTok{  \}}
  \KeywordTok{return}\NormalTok{ fib(x}\DecValTok{-1}\NormalTok{) + fib(x}\DecValTok{-2}\NormalTok{)}
\NormalTok{\}}
\end{Highlighting}
\end{Shaded}

\hypertarget{ux6ce8ux610f}{%
\subsection{注意}\label{ux6ce8ux610f}}

\begin{enumerate}
\tightlist
\item
  main 函数退出的时候,所有goroutine全部暴力退出
\item
  没有程序化的方法让一个goroutine来终止另一个goroutine
\end{enumerate}

\hypertarget{ux5e76ux53d1ux65f6ux949fux670dux52a1ux5668}{%
\section{并发时钟服务器}\label{ux5e76ux53d1ux65f6ux949fux670dux52a1ux5668}}

\hypertarget{clock1}{%
\subsection{clock1}\label{clock1}}

\begin{Shaded}
\begin{Highlighting}[]
\KeywordTok{func}\NormalTok{ main() \{}
\NormalTok{  listener, err := net.Listen(}\StringTok{"tcp"}\NormalTok{, }\StringTok{"localhost:8000"}\NormalTok{)}
  \KeywordTok{if}\NormalTok{ err != }\OtherTok{nil}\NormalTok{ \{}
\NormalTok{    log.Fatal(err)}
\NormalTok{  \}}
  \KeywordTok{for}\NormalTok{ \{}
\NormalTok{    conn, err := listener.Accept()}
    \KeywordTok{if}\NormalTok{ err != }\OtherTok{nil}\NormalTok{ \{}
\NormalTok{      log.Print(err) }\CommentTok{// e.g., connection aborted}
      \KeywordTok{continue}
\NormalTok{    \}}
\NormalTok{    handleConn(conn) }\CommentTok{// handle one connection at a time}
\NormalTok{  \}}
\NormalTok{\}}

\KeywordTok{func}\NormalTok{ handleConn(c net.Conn) \{}
  \KeywordTok{defer}\NormalTok{ c.Close()}
  \KeywordTok{for}\NormalTok{ \{}
\NormalTok{    _, err := io.WriteString(c, time.Now().Format(}\StringTok{"15:04:05}\CharTok{\textbackslash{}n}\StringTok{"}\NormalTok{))}
    \KeywordTok{if}\NormalTok{ err != }\OtherTok{nil}\NormalTok{ \{}
      \KeywordTok{return} \CommentTok{// e.g., client disconnected}
\NormalTok{    \}}
\NormalTok{    time.Sleep(}\DecValTok{1}\NormalTok{ * time.Second)}
\NormalTok{  \}}
\NormalTok{\}}
\CommentTok{//!-}
\end{Highlighting}
\end{Shaded}

当服务器打开后,可以使用

\begin{Shaded}
\begin{Highlighting}[]
\ExtensionTok{nc}\NormalTok{ localhost 8000}
\end{Highlighting}
\end{Shaded}

连接,但是只能允许1个客户端连接。

\hypertarget{netcat1}{%
\subsection{netcat1}\label{netcat1}}

\begin{Shaded}
\begin{Highlighting}[]
\KeywordTok{func}\NormalTok{ main() \{}
\NormalTok{  conn, err := net.Dial(}\StringTok{"tcp"}\NormalTok{, }\StringTok{"localhost:8000"}\NormalTok{)}
  \KeywordTok{if}\NormalTok{ err != }\OtherTok{nil}\NormalTok{ \{}
\NormalTok{    log.Fatal(err)}
\NormalTok{  \}}
  \KeywordTok{defer}\NormalTok{ conn.Close()}
\NormalTok{  mustCopy(os.Stdout, conn)}
\NormalTok{\}}
\KeywordTok{func}\NormalTok{ mustCopy(dst io.Writer, src io.Reader) \{}
  \KeywordTok{if}\NormalTok{ _, err := io.Copy(dst, src); err != }\OtherTok{nil}\NormalTok{ \{}
\NormalTok{    log.Fatal(err)}
\NormalTok{  \}}
\NormalTok{\}}
\end{Highlighting}
\end{Shaded}

\hypertarget{clock2}{%
\subsection{clock2}\label{clock2}}

可以接受多个客户端连接

\begin{Shaded}
\begin{Highlighting}[]
\KeywordTok{func}\NormalTok{ handleConn(c net.Conn) \{}
  \KeywordTok{defer}\NormalTok{ c.Close()}
  \KeywordTok{for}\NormalTok{ \{}
\NormalTok{    _, err := io.WriteString(c, time.Now().Format(}\StringTok{"15:04:05}\CharTok{\textbackslash{}n}\StringTok{"}\NormalTok{))}
    \KeywordTok{if}\NormalTok{ err != }\OtherTok{nil}\NormalTok{ \{}
      \KeywordTok{return} \CommentTok{// e.g., client disconnected}
\NormalTok{    \}}
\NormalTok{    time.Sleep(}\DecValTok{1}\NormalTok{ * time.Second)}
\NormalTok{  \}}
\NormalTok{\}}

\KeywordTok{func}\NormalTok{ main() \{}
\NormalTok{  listener, err := net.Listen(}\StringTok{"tcp"}\NormalTok{, }\StringTok{"localhost:8000"}\NormalTok{)}
  \KeywordTok{if}\NormalTok{ err != }\OtherTok{nil}\NormalTok{ \{}
\NormalTok{    log.Fatal(err)}
\NormalTok{  \}}
  \CommentTok{//!+}
  \KeywordTok{for}\NormalTok{ \{}
\NormalTok{    conn, err := listener.Accept()}
    \KeywordTok{if}\NormalTok{ err != }\OtherTok{nil}\NormalTok{ \{}
\NormalTok{      log.Print(err) }\CommentTok{// e.g., connection aborted}
      \KeywordTok{continue}
\NormalTok{    \}}
    \KeywordTok{go}\NormalTok{ handleConn(conn) }\CommentTok{// handle connections concurrently}
\NormalTok{  \}}
\NormalTok{\}}
\end{Highlighting}
\end{Shaded}

\hypertarget{ux5e76ux53d1ux56deux58f0ux670dux52a1ux5668}{%
\section{并发回声服务器}\label{ux5e76ux53d1ux56deux58f0ux670dux52a1ux5668}}

\hypertarget{reverb1}{%
\subsection{reverb1}\label{reverb1}}

\begin{Shaded}
\begin{Highlighting}[]
\KeywordTok{func}\NormalTok{ echo(c net.Conn, shout }\DataTypeTok{string}\NormalTok{, delay time.Duration) \{}
\NormalTok{  fmt.Fprintln(c, }\StringTok{"}\CharTok{\textbackslash{}t}\StringTok{"}\NormalTok{, strings.ToUpper(shout))}
\NormalTok{  time.Sleep(delay)}
\NormalTok{  fmt.Fprintln(c, }\StringTok{"}\CharTok{\textbackslash{}t}\StringTok{"}\NormalTok{, shout)}
\NormalTok{  time.Sleep(delay)}
\NormalTok{  fmt.Fprintln(c, }\StringTok{"}\CharTok{\textbackslash{}t}\StringTok{"}\NormalTok{, strings.ToLower(shout))}
\NormalTok{\}}

\KeywordTok{func}\NormalTok{ handleConn(c net.Conn) \{}
\NormalTok{  input := bufio.NewScanner(c)}
  \KeywordTok{for}\NormalTok{ input.Scan() \{}
\NormalTok{    echo(c, input.Text(), }\DecValTok{1}\NormalTok{*time.Second)}
\NormalTok{  \}}
  \CommentTok{// }\AlertTok{NOTE}\CommentTok{: ignoring potential errors from input.Err()}
\NormalTok{  c.Close()}
\NormalTok{\}}

\KeywordTok{func}\NormalTok{ main() \{}
\NormalTok{  l, err := net.Listen(}\StringTok{"tcp"}\NormalTok{, }\StringTok{"localhost:8000"}\NormalTok{)}
  \KeywordTok{if}\NormalTok{ err != }\OtherTok{nil}\NormalTok{ \{}
\NormalTok{    log.Fatal(err)}
\NormalTok{  \}}
  \KeywordTok{for}\NormalTok{ \{}
\NormalTok{    conn, err := l.Accept()}
    \KeywordTok{if}\NormalTok{ err != }\OtherTok{nil}\NormalTok{ \{}
\NormalTok{      log.Print(err) }\CommentTok{// e.g., connection aborted}
      \KeywordTok{continue}
\NormalTok{    \}}
    \KeywordTok{go}\NormalTok{ handleConn(conn)}
\NormalTok{  \}}
\NormalTok{\}}
\end{Highlighting}
\end{Shaded}

注意所有的返回是按顺序的,

\begin{Shaded}
\begin{Highlighting}[]
\ExtensionTok{some}
    \ExtensionTok{SOME}
\ExtensionTok{GOOD}
    \ExtensionTok{some}
    \ExtensionTok{some}
    \ExtensionTok{GOOD}
    \ExtensionTok{good}
    \ExtensionTok{good}
\end{Highlighting}
\end{Shaded}

\hypertarget{reverb2}{%
\subsection{reverb2}\label{reverb2}}

\begin{Shaded}
\begin{Highlighting}[]
\KeywordTok{func}\NormalTok{ echo(c net.Conn, shout }\DataTypeTok{string}\NormalTok{, delay time.Duration) \{}
\NormalTok{  fmt.Fprintln(c, }\StringTok{"}\CharTok{\textbackslash{}t}\StringTok{"}\NormalTok{, strings.ToUpper(shout))}
\NormalTok{  time.Sleep(delay)}
\NormalTok{  fmt.Fprintln(c, }\StringTok{"}\CharTok{\textbackslash{}t}\StringTok{"}\NormalTok{, shout)}
\NormalTok{  time.Sleep(delay)}
\NormalTok{  fmt.Fprintln(c, }\StringTok{"}\CharTok{\textbackslash{}t}\StringTok{"}\NormalTok{, strings.ToLower(shout))}
\NormalTok{\}}

\KeywordTok{func}\NormalTok{ handleConn(c net.Conn) \{}
\NormalTok{  input := bufio.NewScanner(c)}
  \KeywordTok{for}\NormalTok{ input.Scan() \{}
    \KeywordTok{go}\NormalTok{ echo(c, input.Text(), }\DecValTok{1}\NormalTok{*time.Second)}
\NormalTok{  \}}
  \CommentTok{// }\AlertTok{NOTE}\CommentTok{: ignoring potential errors from input.Err()}
\NormalTok{  c.Close()}
\NormalTok{\}}

\KeywordTok{func}\NormalTok{ main() \{}
\NormalTok{  l, err := net.Listen(}\StringTok{"tcp"}\NormalTok{, }\StringTok{"localhost:8000"}\NormalTok{)}
  \KeywordTok{if}\NormalTok{ err != }\OtherTok{nil}\NormalTok{ \{}
\NormalTok{    log.Fatal(err)}
\NormalTok{  \}}
  \KeywordTok{for}\NormalTok{ \{}
\NormalTok{    conn, err := l.Accept()}
    \KeywordTok{if}\NormalTok{ err != }\OtherTok{nil}\NormalTok{ \{}
\NormalTok{      log.Print(err) }\CommentTok{// e.g., connection aborted}
      \KeywordTok{continue}
\NormalTok{    \}}
    \KeywordTok{go}\NormalTok{ handleConn(conn)}
\NormalTok{  \}}
\NormalTok{\}}
\end{Highlighting}
\end{Shaded}

返回是乱序的,因为并发返回

\begin{Shaded}
\begin{Highlighting}[]
\ExtensionTok{some}
   \ExtensionTok{SOME}
\ExtensionTok{good}
   \ExtensionTok{GOOD}
   \ExtensionTok{some}
   \ExtensionTok{good}
   \ExtensionTok{some}
   \ExtensionTok{good}
\end{Highlighting}
\end{Shaded}

\hypertarget{ux901aux9053}{%
\section{通道}\label{ux901aux9053}}

\begin{enumerate}
\tightlist
\item
  用来在 goroutine 之间进行通信用的。
\item
  通道是引用类型。
\item
  通道类型之间可以比较 \(==\), 通道的0值为nil
\end{enumerate}

\begin{Shaded}
\begin{Highlighting}[]
\NormalTok{ch := }\BuiltInTok{make}\NormalTok{(}\KeywordTok{chan} \DataTypeTok{int}\NormalTok{)}
\NormalTok{ch <- x}
\NormalTok{x = <- ch}
\NormalTok{<-ch}
\BuiltInTok{close}\NormalTok{(ch)}
\end{Highlighting}
\end{Shaded}

\begin{enumerate}
\tightlist
\item
  往一个关闭的通道发送数据,将引发宕机。
\item
  从一个关闭的通道获取数据,会不断的获取数据。直到获取完毕所有数据。然后,之后的操作立即返回nil。
\item
  重复关闭一个通道也会引发宕机。
\end{enumerate}

\begin{Shaded}
\begin{Highlighting}[]
\NormalTok{ch = }\BuiltInTok{make}\NormalTok{(}\KeywordTok{chan} \DataTypeTok{int}\NormalTok{)     }\CommentTok{// 无缓冲通道}
\NormalTok{ch = }\BuiltInTok{make}\NormalTok{(}\KeywordTok{chan} \DataTypeTok{int}\NormalTok{, }\DecValTok{0}\NormalTok{)  }\CommentTok{// 无缓冲通道}
\NormalTok{ch = }\BuiltInTok{make}\NormalTok{(}\KeywordTok{chan} \DataTypeTok{int}\NormalTok{, }\DecValTok{1}\NormalTok{)  }\CommentTok{// 容量为1的通道}
\NormalTok{ch = }\BuiltInTok{make}\NormalTok{(}\KeywordTok{chan}\NormalTok{ init, }\DecValTok{3}\NormalTok{) }\CommentTok{// 容量为3的通道}
\end{Highlighting}
\end{Shaded}

无缓冲通道和容量为1的通道是不一样的两种通道。

\begin{Shaded}
\begin{Highlighting}[]
\NormalTok{ch = }\BuiltInTok{make}\NormalTok{(}\KeywordTok{chan} \DataTypeTok{int}\NormalTok{, }\DecValTok{1}\NormalTok{)  }\CommentTok{// 容量为1的通道}
\NormalTok{ch <- }\DecValTok{10}
\NormalTok{fmt.Println(}\StringTok{"something"}\NormalTok{) }\CommentTok{// 在ch<-10之后立即执行}
\CommentTok{// goroutine 1}
\NormalTok{ch = }\BuiltInTok{make}\NormalTok{(}\KeywordTok{chan} \DataTypeTok{int}\NormalTok{, }\DecValTok{0}\NormalTok{)  }\CommentTok{// 容量为0的通道}
\NormalTok{ch <- }\DecValTok{10} \CommentTok{// 阻塞住}
\NormalTok{fmt.Println(}\StringTok{"something"}\NormalTok{) }\CommentTok{// 不会立即执行}

\CommentTok{// goroutine 2}
\NormalTok{<-ch }\CommentTok{// 当这句话被执行的时候goroutine 1中的阻塞才会被消除。}
\CommentTok{//}
\end{Highlighting}
\end{Shaded}

第一个向无缓冲通道里面写入或提取输出的goroutine都会阻塞,直到第二个goroutine去提出或者写入数据。不同于缓冲为1的通道。这也是为什么无缓冲通道可以用来同步两个goroutine的原因。

\hypertarget{ux65e0ux7f13ux51b2ux901aux9053}{%
\subsection{无缓冲通道}\label{ux65e0ux7f13ux51b2ux901aux9053}}

\begin{enumerate}
\tightlist
\item
  无缓冲通道也叫做同步通道。
\end{enumerate}

\begin{Shaded}
\begin{Highlighting}[]
\KeywordTok{func}\NormalTok{ main() \{}
\NormalTok{  conn, err := net.Dial(}\StringTok{"tcp"}\NormalTok{, }\StringTok{"localhost:8000"}\NormalTok{)}
  \KeywordTok{if}\NormalTok{ err != }\OtherTok{nil}\NormalTok{ \{}
\NormalTok{    log.Fatal(err)}
\NormalTok{  \}}
\NormalTok{  done := }\BuiltInTok{make}\NormalTok{(}\KeywordTok{chan} \KeywordTok{struct}\NormalTok{\{\})}
  \KeywordTok{go} \KeywordTok{func}\NormalTok{() \{}
\NormalTok{    io.Copy(os.Stdout, conn) }\CommentTok{// }\AlertTok{NOTE}\CommentTok{: ignoring errors}
\NormalTok{    log.Println(}\StringTok{"done"}\NormalTok{)}
\NormalTok{    done <- }\KeywordTok{struct}\NormalTok{\{\}\{\} }\CommentTok{// signal the main goroutine}
\NormalTok{  \}()}
\NormalTok{  mustCopy(conn, os.Stdin)}\CommentTok{//注意主程序只执行到这一步,就卡住了。因为要读取键盘输入。}
\NormalTok{  conn.Close()}
\NormalTok{  <-done }\CommentTok{// wait for background goroutine to finish}
\NormalTok{\}}

\KeywordTok{func}\NormalTok{ mustCopy(dst io.Writer, src io.Reader) \{}
  \KeywordTok{if}\NormalTok{ _, err := io.Copy(dst, src); err != }\OtherTok{nil}\NormalTok{ \{}
\NormalTok{    log.Fatal(err)}
\NormalTok{  \}}
\NormalTok{\}}
\end{Highlighting}
\end{Shaded}

等待计算结果

\begin{Shaded}
\begin{Highlighting}[]
\KeywordTok{package}\NormalTok{ main}
\KeywordTok{import}\NormalTok{ (}
  \StringTok{"fmt"}
  \StringTok{"time"}
\NormalTok{)}
\KeywordTok{func}\NormalTok{ main() \{}
\NormalTok{  result := }\DecValTok{0}
\NormalTok{  done := }\BuiltInTok{make}\NormalTok{(}\KeywordTok{chan} \KeywordTok{struct}\NormalTok{\{\})}
  \KeywordTok{go} \KeywordTok{func}\NormalTok{() \{}
    \KeywordTok{for}\NormalTok{ i := }\DecValTok{1}\NormalTok{; i < }\DecValTok{10}\NormalTok{; i++ \{}
\NormalTok{      result = i + result}
\NormalTok{      time.Sleep(}\DecValTok{1000}\NormalTok{ * time.Millisecond)}
\NormalTok{      fmt.Println(}\StringTok{"....."}\NormalTok{)}
\NormalTok{    \}}
\NormalTok{    done <- }\KeywordTok{struct}\NormalTok{\{\}\{\}}
\NormalTok{  \}()}
\NormalTok{  <-done}
\NormalTok{  fmt.Println(result)}
\NormalTok{\}}
\end{Highlighting}
\end{Shaded}

\hypertarget{ux7ba1ux9053}{%
\subsection{管道}\label{ux7ba1ux9053}}

\hypertarget{pipeline1}{%
\subsubsection{pipeline1}\label{pipeline1}}

\begin{Shaded}
\begin{Highlighting}[]
\KeywordTok{func}\NormalTok{ main() \{}
\NormalTok{  naturals := }\BuiltInTok{make}\NormalTok{(}\KeywordTok{chan} \DataTypeTok{int}\NormalTok{)}
\NormalTok{  squares := }\BuiltInTok{make}\NormalTok{(}\KeywordTok{chan} \DataTypeTok{int}\NormalTok{)}

  \CommentTok{// Counter}
  \KeywordTok{go} \KeywordTok{func}\NormalTok{() \{}
    \KeywordTok{for}\NormalTok{ x := }\DecValTok{0}\NormalTok{; ; x++ \{}
\NormalTok{      naturals <- x}
\NormalTok{    \}}
\NormalTok{  \}()}

  \CommentTok{// Squarer}
  \KeywordTok{go} \KeywordTok{func}\NormalTok{() \{}
    \KeywordTok{for}\NormalTok{ \{}
\NormalTok{      x := <-naturals}
\NormalTok{      squares <- x * x}
\NormalTok{    \}}
\NormalTok{  \}()}

  \CommentTok{// Printer (in main goroutine)}
  \KeywordTok{for}\NormalTok{ \{}
\NormalTok{    fmt.Println(<-squares)}
\NormalTok{  \}}
\NormalTok{\}}
\end{Highlighting}
\end{Shaded}

\hypertarget{pipeline2}{%
\subsubsection{pipeline2}\label{pipeline2}}

\begin{Shaded}
\begin{Highlighting}[]
\KeywordTok{func}\NormalTok{ main() \{}
\NormalTok{  naturals := }\BuiltInTok{make}\NormalTok{(}\KeywordTok{chan} \DataTypeTok{int}\NormalTok{)}
\NormalTok{  squares := }\BuiltInTok{make}\NormalTok{(}\KeywordTok{chan} \DataTypeTok{int}\NormalTok{)}

  \CommentTok{// Counter}
  \KeywordTok{go} \KeywordTok{func}\NormalTok{() \{}
    \KeywordTok{for}\NormalTok{ x := }\DecValTok{0}\NormalTok{; x < }\DecValTok{100}\NormalTok{; x++ \{}
\NormalTok{      naturals <- x}
\NormalTok{    \}}
    \BuiltInTok{close}\NormalTok{(naturals)}
\NormalTok{  \}()}

  \CommentTok{// Squarer}
  \KeywordTok{go} \KeywordTok{func}\NormalTok{() \{}
    \KeywordTok{for}\NormalTok{ x := }\KeywordTok{range}\NormalTok{ naturals \{}
\NormalTok{      squares <- x * x}
\NormalTok{    \}}
    \BuiltInTok{close}\NormalTok{(squares)}
\NormalTok{  \}()}

  \CommentTok{// Printer (in main goroutine)}
  \KeywordTok{for}\NormalTok{ x := }\KeywordTok{range}\NormalTok{ squares \{}
\NormalTok{    fmt.Println(x)}
\NormalTok{  \}}
\NormalTok{\}}
\end{Highlighting}
\end{Shaded}

\begin{enumerate}
\tightlist
\item
  关闭每一个通道不是必须的。
\item
  通道的关闭是通过是否可以访问来确定是否资源回收。而不是是否关闭
\end{enumerate}

\hypertarget{ux5355ux5411ux901aux9053ux7c7bux578b}{%
\subsection{单向通道类型}\label{ux5355ux5411ux901aux9053ux7c7bux578b}}

\begin{Shaded}
\begin{Highlighting}[]
\KeywordTok{func}\NormalTok{ counter(out }\KeywordTok{chan}\NormalTok{<- }\DataTypeTok{int}\NormalTok{) \{}
  \KeywordTok{for}\NormalTok{ x := }\DecValTok{0}\NormalTok{; x < }\DecValTok{100}\NormalTok{; x++ \{}
\NormalTok{    out <- x}
\NormalTok{  \}}
  \BuiltInTok{close}\NormalTok{(out)}
\NormalTok{\}}

\KeywordTok{func}\NormalTok{ squarer(out }\KeywordTok{chan}\NormalTok{<- }\DataTypeTok{int}\NormalTok{, in <-}\KeywordTok{chan} \DataTypeTok{int}\NormalTok{) \{}
  \KeywordTok{for}\NormalTok{ v := }\KeywordTok{range}\NormalTok{ in \{}
\NormalTok{    out <- v * v}
\NormalTok{  \}}
  \BuiltInTok{close}\NormalTok{(out)}
\NormalTok{\}}

\KeywordTok{func}\NormalTok{ printer(in <-}\KeywordTok{chan} \DataTypeTok{int}\NormalTok{) \{}
  \KeywordTok{for}\NormalTok{ v := }\KeywordTok{range}\NormalTok{ in \{}
\NormalTok{    fmt.Println(v)}
\NormalTok{  \}}
\NormalTok{\}}

\KeywordTok{func}\NormalTok{ main() \{}
\NormalTok{  naturals := }\BuiltInTok{make}\NormalTok{(}\KeywordTok{chan} \DataTypeTok{int}\NormalTok{)}
\NormalTok{  squares := }\BuiltInTok{make}\NormalTok{(}\KeywordTok{chan} \DataTypeTok{int}\NormalTok{)}

  \KeywordTok{go}\NormalTok{ counter(naturals)}
  \KeywordTok{go}\NormalTok{ squarer(squares, naturals)}
\NormalTok{  printer(squares)}
\NormalTok{\}}
\end{Highlighting}
\end{Shaded}

\hypertarget{ux7f13ux51b2ux901aux9053}{%
\subsection{缓冲通道}\label{ux7f13ux51b2ux901aux9053}}

\begin{Shaded}
\begin{Highlighting}[]
\NormalTok{ch := }\BuiltInTok{make}\NormalTok{(}\KeywordTok{chan} \DataTypeTok{string}\NormalTok{, }\DecValTok{3}\NormalTok{)}
\NormalTok{fmt.Println(}\BuiltInTok{cap}\NormalTok{(ch)) }\CommentTok{// 3}
\NormalTok{fmt.Println(}\BuiltInTok{len}\NormalTok{(ch)) }\CommentTok{// 2 当前有多少个元素在里面}
\end{Highlighting}
\end{Shaded}

\begin{enumerate}
\tightlist
\item
  注意不能把管道当做队列来用。因为使用队列来用稍有不甚,会导致程序卡住。
\end{enumerate}

\begin{Shaded}
\begin{Highlighting}[]
\KeywordTok{func}\NormalTok{ mirroredQuery() }\DataTypeTok{string}\NormalTok{ \{}
\NormalTok{  responses := }\BuiltInTok{make}\NormalTok{(}\KeywordTok{chan} \DataTypeTok{string}\NormalTok{, }\DecValTok{3}\NormalTok{)}
  \KeywordTok{go} \KeywordTok{func}\NormalTok{() \{responses <- request(}\StringTok{"asia.gopl.io"}\NormalTok{)\} ()}
  \KeywordTok{go} \KeywordTok{func}\NormalTok{() \{responses <- request(}\StringTok{"europe.gopl.io"}\NormalTok{)\} ()}
  \KeywordTok{go} \KeywordTok{func}\NormalTok{() \{responses <- request(}\StringTok{"americas.gopl.io"}\NormalTok{)\} ()}
  \KeywordTok{return}\NormalTok{ <-responses}
\NormalTok{\}}

\KeywordTok{func}\NormalTok{ request(hostname }\DataTypeTok{string}\NormalTok{) (response }\DataTypeTok{string}\NormalTok{) \{}\CommentTok{/* ... */}\NormalTok{\}}
\end{Highlighting}
\end{Shaded}

\begin{enumerate}
\tightlist
\item
  如果采用无缓冲的chan,那么慢的两个goroutine就会卡住。不会被释放掉,因为这两个goroutine阻塞住了,无法运行。
\item
  具体是怎么泄露的? fix
\end{enumerate}

\hypertarget{ux5e76ux884cux5faaux73af}{%
\section{并行循环}\label{ux5e76ux884cux5faaux73af}}

无并行程序写法

\begin{Shaded}
\begin{Highlighting}[]
\KeywordTok{func}\NormalTok{ makeThumbnails(filenames []}\DataTypeTok{string}\NormalTok{) \{}
  \KeywordTok{for}\NormalTok{ _, f := }\KeywordTok{range}\NormalTok{ filenames \{}
    \KeywordTok{if}\NormalTok{ _, err := thumbnail.ImageFile(f); err != }\OtherTok{nil}\NormalTok{ \{}
\NormalTok{      log.Println(err)}
\NormalTok{    \}}
\NormalTok{  \}}
\NormalTok{\}}
\end{Highlighting}
\end{Shaded}

并行写法

\begin{Shaded}
\begin{Highlighting}[]
\KeywordTok{func}\NormalTok{ makeThumbnails(filenames []}\DataTypeTok{string}\NormalTok{) \{}
  \KeywordTok{for}\NormalTok{ _, f := }\KeywordTok{range}\NormalTok{ filenames \{}
    \KeywordTok{go}\NormalTok{ thumbnail.ImageFile(f) }\CommentTok{// 这个地方存在问题,程序又错误!!!}
\NormalTok{  \}}
\NormalTok{\}}
\end{Highlighting}
\end{Shaded}

上面的程序有两个地方存在问题。

\begin{enumerate}
\tightlist
\item
  f的问题
\item
  程序运行之后瞬间推出。并没有任何函数再等待goroutine运行完毕。
\end{enumerate}

\begin{Shaded}
\begin{Highlighting}[]
\KeywordTok{func}\NormalTok{ makeThumbnails(filenames []}\DataTypeTok{string}\NormalTok{)\{}
\NormalTok{  ch := }\BuiltInTok{make}\NormalTok{(}\KeywordTok{chan} \KeywordTok{struct}\NormalTok{\{\})}
  \KeywordTok{for}\NormalTok{ _, f : }\KeywordTok{range}\NormalTok{ filenames \{}
    \KeywordTok{go} \KeywordTok{func}\NormalTok{(f }\DataTypeTok{string}\NormalTok{) \{}
\NormalTok{      thumbnail.ImageFile(f)}
\NormalTok{      ch <-}\KeywordTok{struct}\NormalTok{\{\}\{\}}
\NormalTok{    \}(f)}
\NormalTok{  \}}

  \KeywordTok{for} \KeywordTok{range}\NormalTok{ filenames \{}
\NormalTok{    <-ch}
\NormalTok{  \}}
\NormalTok{\}}
\end{Highlighting}
\end{Shaded}

处理返回值

\begin{Shaded}
\begin{Highlighting}[]
\KeywordTok{func}\NormalTok{ makeThumbnails(filenames []}\DataTypeTok{string}\NormalTok{) }\DataTypeTok{error}\NormalTok{\{}
\NormalTok{  errors := }\BuiltInTok{make}\NormalTok{(}\KeywordTok{chan} \DataTypeTok{error}\NormalTok{)}
  \KeywordTok{for}\NormalTok{ _, f := }\KeywordTok{range}\NormalTok{ filenames \{}
\NormalTok{    _, err := thumbnail.ImageFile(f)}
\NormalTok{    errors <- err}
\NormalTok{  \}(f)}

  \KeywordTok{for} \KeywordTok{range}\NormalTok{ filenames \{}
    \KeywordTok{if}\NormalTok{ err := <-errors; err != }\OtherTok{nil}\NormalTok{ \{}
      \KeywordTok{return}\NormalTok{ err }\CommentTok{// 这里不正确,goroutine泄露危险}
\NormalTok{    \}}
\NormalTok{  \}}

  \KeywordTok{return} \OtherTok{nil}
\NormalTok{\}}
\end{Highlighting}
\end{Shaded}

Q:为什么会造成goroutine协奏,goroutine泄露的原理是什么?

解决方案,是开辟一个和filenames一样长的通道。

\begin{Shaded}
\begin{Highlighting}[]
\KeywordTok{func}\NormalTok{ makeThumbnails5(filenames []}\DataTypeTok{string}\NormalTok{) (thumbfiles []}\DataTypeTok{string}\NormalTok{, err }\DataTypeTok{error}\NormalTok{) \{}
  \KeywordTok{type}\NormalTok{ item }\KeywordTok{struct}\NormalTok{ \{}
\NormalTok{    thumbfile }\DataTypeTok{string}
\NormalTok{    err       }\DataTypeTok{error}
\NormalTok{  \}}
\NormalTok{  ch := }\BuiltInTok{make}\NormalTok{(}\KeywordTok{chan}\NormalTok{ item, }\BuiltInTok{len}\NormalTok{(filenames))}
  \KeywordTok{for}\NormalTok{ _, f := }\KeywordTok{range}\NormalTok{ filenames \{}
    \KeywordTok{go} \KeywordTok{func}\NormalTok{(f }\DataTypeTok{string}\NormalTok{) \{}
      \KeywordTok{var}\NormalTok{ it item}
\NormalTok{      it.thumbfile, it.err = thumbnail.ImageFile(f)}
\NormalTok{      ch <- it}
\NormalTok{      \}(f)}
\NormalTok{  \}}
  \KeywordTok{for} \KeywordTok{range}\NormalTok{ filenames \{}
\NormalTok{    it := <-ch}
    \KeywordTok{if}\NormalTok{ it.err != }\OtherTok{nil}\NormalTok{ \{}
      \KeywordTok{return} \OtherTok{nil}\NormalTok{, it.err}
\NormalTok{    \}}
\NormalTok{    thumbfiles = }\BuiltInTok{append}\NormalTok{(thumbfiles, it.thumbfile)}
\NormalTok{  \}}
  \KeywordTok{return}\NormalTok{ thumbfiles, }\OtherTok{nil}
\NormalTok{\}}
\end{Highlighting}
\end{Shaded}

一个更好的解决方案

\begin{Shaded}
\begin{Highlighting}[]
\KeywordTok{func}\NormalTok{ makeThumbnails6(filenames <-}\KeywordTok{chan} \DataTypeTok{string}\NormalTok{) }\DataTypeTok{int64}\NormalTok{ \{}
\NormalTok{  sizes := }\BuiltInTok{make}\NormalTok{(}\KeywordTok{chan} \DataTypeTok{int64}\NormalTok{)}
  \KeywordTok{var}\NormalTok{ wg sync.WaitGroup }\CommentTok{// number of working goroutines}
  \KeywordTok{for}\NormalTok{ f := }\KeywordTok{range}\NormalTok{ filenames \{}
\NormalTok{    wg.Add(}\DecValTok{1}\NormalTok{)}
    \CommentTok{// worker}
    \KeywordTok{go} \KeywordTok{func}\NormalTok{(f }\DataTypeTok{string}\NormalTok{) \{}
      \KeywordTok{defer}\NormalTok{ wg.Done()}
\NormalTok{      thumb, err := thumbnail.ImageFile(f)}
      \KeywordTok{if}\NormalTok{ err != }\OtherTok{nil}\NormalTok{ \{}
\NormalTok{        log.Println(err)}
        \KeywordTok{return}\NormalTok{ \}}
\NormalTok{      info, _ := os.Stat(thumb) }\CommentTok{// OK to ignore error}
\NormalTok{      sizes <- info.Size()}
\NormalTok{    \}(f)}
\NormalTok{  \}}
  \CommentTok{// closer}
  \KeywordTok{go} \KeywordTok{func}\NormalTok{() \{}
\NormalTok{    wg.Wait()}
    \BuiltInTok{close}\NormalTok{(sizes)}
\NormalTok{  \}()}
  \KeywordTok{var}\NormalTok{ total }\DataTypeTok{int64}
  \KeywordTok{for}\NormalTok{ size := }\KeywordTok{range}\NormalTok{ sizes \{}
\NormalTok{    total += size}
\NormalTok{  \}}
  \KeywordTok{return}\NormalTok{ total}
\NormalTok{\}}
\end{Highlighting}
\end{Shaded}

\hypertarget{ux5e76ux53d1ux7684webux722cux866b}{%
\section{并发的Web爬虫}\label{ux5e76ux53d1ux7684webux722cux866b}}

\hypertarget{ux7b2cux4e00ux7248}{%
\subsection{第一版}\label{ux7b2cux4e00ux7248}}

\begin{Shaded}
\begin{Highlighting}[]
\KeywordTok{func}\NormalTok{ crawl(url }\DataTypeTok{string}\NormalTok{) []}\DataTypeTok{string}\NormalTok{ \{}
\NormalTok{  fmt.Println(url)}
\NormalTok{  list, err := links.Extract(url)}
  \KeywordTok{if}\NormalTok{ err != }\OtherTok{nil}\NormalTok{ \{}
\NormalTok{    log.Print(err)}
\NormalTok{  \}}
  \KeywordTok{return}\NormalTok{ list}
\NormalTok{\}}
\end{Highlighting}
\end{Shaded}

\begin{Shaded}
\begin{Highlighting}[]
\KeywordTok{func}\NormalTok{ main() \{}
\NormalTok{  worklist := }\BuiltInTok{make}\NormalTok{(}\KeywordTok{chan}\NormalTok{ []}\DataTypeTok{string}\NormalTok{)}
  \CommentTok{// Start with the command-line arguments.}
  \KeywordTok{go} \KeywordTok{func}\NormalTok{() \{ worklist <- os.Args[}\DecValTok{1}\NormalTok{:] \}()}
  \CommentTok{// Crawl the web concurrently.}
\NormalTok{  seen := }\BuiltInTok{make}\NormalTok{(}\KeywordTok{map}\NormalTok{[}\DataTypeTok{string}\NormalTok{]}\DataTypeTok{bool}\NormalTok{)}
  \KeywordTok{for}\NormalTok{ list := }\KeywordTok{range}\NormalTok{ worklist \{ }\CommentTok{// 注意,不会结束的原因在这里。因为阻塞住了。当worklist中的内容为空时。阻塞住。}
    \KeywordTok{for}\NormalTok{ _, link := }\KeywordTok{range}\NormalTok{ list \{}
      \KeywordTok{if}\NormalTok{ !seen[link] \{}
\NormalTok{        seen[link] = }\OtherTok{true}
        \KeywordTok{go} \KeywordTok{func}\NormalTok{(link }\DataTypeTok{string}\NormalTok{)\{}
\NormalTok{          worklist <- crawl(link)}
\NormalTok{        \}(link)}
\NormalTok{      \}}
\NormalTok{    \}}
\NormalTok{  \}}
\NormalTok{\}}
\end{Highlighting}
\end{Shaded}

问题:

\begin{enumerate}
\tightlist
\item
  打开太多的链接。并发度太高,导致文件描述符不够用
\item
  程序并没有终止。
\end{enumerate}

\hypertarget{ux89e3ux51b3ux94feux63a5ux592aux591aux95eeux9898}{%
\subsection{解决链接太多问题}\label{ux89e3ux51b3ux94feux63a5ux592aux591aux95eeux9898}}

\begin{Shaded}
\begin{Highlighting}[]
\KeywordTok{var}\NormalTok{ tokens = }\BuiltInTok{make}\NormalTok{(}\KeywordTok{chan} \KeywordTok{struct}\NormalTok{\{\}, }\DecValTok{20}\NormalTok{)}
\KeywordTok{func}\NormalTok{ crawl(url }\DataTypeTok{string}\NormalTok{) []}\DataTypeTok{string}\NormalTok{ \{}
\NormalTok{  fmt.Println(url)}
\NormalTok{  tokens <- }\KeywordTok{struct}\NormalTok{\{\}\{\} }\CommentTok{// acquire a token}
\NormalTok{  list, err := links.Extract(url) }\CommentTok{// 最多只有20个go线程在获取url对应的所有链接。}
\NormalTok{  <-tokens }\CommentTok{// release the token}
  \KeywordTok{if}\NormalTok{ err != }\OtherTok{nil}\NormalTok{ \{}
\NormalTok{    log.Print(err)}
\NormalTok{  \}}
  \KeywordTok{return}\NormalTok{ list}
\NormalTok{\}}
\end{Highlighting}
\end{Shaded}

\hypertarget{ux89e3ux51b3ux7ec8ux6b62ux95eeux9898}{%
\subsection{解决终止问题}\label{ux89e3ux51b3ux7ec8ux6b62ux95eeux9898}}

\begin{Shaded}
\begin{Highlighting}[]
\KeywordTok{func}\NormalTok{ main() \{}
\NormalTok{  worklist := }\BuiltInTok{make}\NormalTok{(}\KeywordTok{chan}\NormalTok{ []}\DataTypeTok{string}\NormalTok{)}
  \KeywordTok{var}\NormalTok{ n }\DataTypeTok{int} \CommentTok{// number of pending sends to worklist}
  \CommentTok{// Start with the command-line arguments.}
\NormalTok{  n++}
  \KeywordTok{go} \KeywordTok{func}\NormalTok{() \{ worklist <- os.Args[}\DecValTok{1}\NormalTok{:] \}()}
  \CommentTok{// Crawl the web concurrently.}
\NormalTok{  seen := }\BuiltInTok{make}\NormalTok{(}\KeywordTok{map}\NormalTok{[}\DataTypeTok{string}\NormalTok{]}\DataTypeTok{bool}\NormalTok{)}
  \KeywordTok{for}\NormalTok{ ; n > }\DecValTok{0}\NormalTok{; n-- \{}
\NormalTok{    list := <-worklist}
    \KeywordTok{for}\NormalTok{ _, link := }\KeywordTok{range}\NormalTok{ list \{}
      \KeywordTok{if}\NormalTok{ !seen[link] \{}
\NormalTok{        seen[link] = }\OtherTok{true}
\NormalTok{        n++}
        \KeywordTok{go} \KeywordTok{func}\NormalTok{(link }\DataTypeTok{string}\NormalTok{) \{}
\NormalTok{          worklist <- crawl(link)}
\NormalTok{        \}(link)}
\NormalTok{      \}}
\NormalTok{    \}}
\NormalTok{  \}}
\NormalTok{\}}
\end{Highlighting}
\end{Shaded}

\hypertarget{ux4e00ux4e2aux66ffux4ee3ux65b9ux6848}{%
\subsection{一个替代方案}\label{ux4e00ux4e2aux66ffux4ee3ux65b9ux6848}}

\textbf{解决连接数过多, 没有解决终止问题?}

\begin{Shaded}
\begin{Highlighting}[]
\KeywordTok{func}\NormalTok{ main() \{}
\NormalTok{  worklist := }\BuiltInTok{make}\NormalTok{(}\KeywordTok{chan}\NormalTok{ []}\DataTypeTok{string}\NormalTok{)  }\CommentTok{// lists of URLs, may have duplicates}
\NormalTok{  unseenLinks := }\BuiltInTok{make}\NormalTok{(}\KeywordTok{chan} \DataTypeTok{string}\NormalTok{) }\CommentTok{// de-duplicated URLs}
  \CommentTok{// Add command-line arguments to worklist.}
  \KeywordTok{go} \KeywordTok{func}\NormalTok{() \{ worklist <- os.Args[}\DecValTok{1}\NormalTok{:] \}()}
  \CommentTok{// Create 20 crawler goroutines to fetch each unseen link.}
  \KeywordTok{for}\NormalTok{ i := }\DecValTok{0}\NormalTok{; i < }\DecValTok{20}\NormalTok{; i++ \{}
    \KeywordTok{go} \KeywordTok{func}\NormalTok{() \{}
      \KeywordTok{for}\NormalTok{ link := }\KeywordTok{range}\NormalTok{ unseenLinks \{}
\NormalTok{        foundLinks := crawl(link)}
        \KeywordTok{go} \KeywordTok{func}\NormalTok{() \{ worklist <- foundLinks \}()}
\NormalTok{      \}}
\NormalTok{    \}() }
\NormalTok{  \}}
  \CommentTok{// The main goroutine de-duplicates worklist items}
  \CommentTok{// and sends the unseen ones to the crawlers.}
\NormalTok{  seen := }\BuiltInTok{make}\NormalTok{(}\KeywordTok{map}\NormalTok{[}\DataTypeTok{string}\NormalTok{]}\DataTypeTok{bool}\NormalTok{)}
  \KeywordTok{for}\NormalTok{ list := }\KeywordTok{range}\NormalTok{ worklist \{}
    \KeywordTok{for}\NormalTok{ _, link := }\KeywordTok{range}\NormalTok{ list \{}
      \KeywordTok{if}\NormalTok{ !seen[link] \{}
\NormalTok{        seen[link] = }\OtherTok{true}
\NormalTok{        unseenLinks <- link}
\NormalTok{      \}}
\NormalTok{    \}}
\NormalTok{  \}}
\NormalTok{\}}
\end{Highlighting}
\end{Shaded}

\hypertarget{ux591aux8defselect}{%
\section{多路select}\label{ux591aux8defselect}}

\hypertarget{ux5012ux8ba1ux65f6ux7a0bux5e8f}{%
\subsection{倒计时程序}\label{ux5012ux8ba1ux65f6ux7a0bux5e8f}}

倒计时发射程序

\begin{Shaded}
\begin{Highlighting}[]
\KeywordTok{func}\NormalTok{ main() \{}
\NormalTok{  fmt.Println(}\StringTok{"Commencing countdown."}\NormalTok{)}
\NormalTok{  tick := time.Tick(}\DecValTok{1}\NormalTok{ * time.Second)}
  \KeywordTok{for}\NormalTok{ countdown := }\DecValTok{10}\NormalTok{; countdown > }\DecValTok{0}\NormalTok{; countdown-- \{}
\NormalTok{    fmt.Println(countdown)}
\NormalTok{    <-tick}
\NormalTok{  \}}
\NormalTok{  launch()}
\NormalTok{\}}
\end{Highlighting}
\end{Shaded}

取消发射程序

\begin{Shaded}
\begin{Highlighting}[]
\NormalTok{abort := }\BuiltInTok{make}\NormalTok{(}\KeywordTok{chan} \KeywordTok{struct}\NormalTok{\{\})}
\KeywordTok{go} \KeywordTok{func}\NormalTok{() \{}
\NormalTok{  os.Stdin.Read(}\BuiltInTok{make}\NormalTok{([]}\DataTypeTok{byte}\NormalTok{, }\DecValTok{1}\NormalTok{)) }\CommentTok{// read a single byte}
\NormalTok{  abort <- }\KeywordTok{struct}\NormalTok{\{\}\{\}}
\NormalTok{\}()}
\end{Highlighting}
\end{Shaded}

如何同时接收并处理tick和abort消息?select

\begin{Shaded}
\begin{Highlighting}[]
\KeywordTok{select}\NormalTok{ \{}
     \KeywordTok{case}\NormalTok{ <-ch1:}
         \CommentTok{// ...}
     \KeywordTok{case}\NormalTok{ x := <-ch2:}
         \CommentTok{// ...use x...}
     \KeywordTok{case}\NormalTok{ ch3 <- y:}
         \CommentTok{// ...}
     \KeywordTok{default}\NormalTok{:}
  \CommentTok{// ...}
\NormalTok{\}}
\end{Highlighting}
\end{Shaded}

最终实现

\begin{Shaded}
\begin{Highlighting}[]
\KeywordTok{func}\NormalTok{ main() \{}
  \CommentTok{// ...create abort channel...}
\NormalTok{  fmt.Println(}\StringTok{"Commencing countdown.  Press return to abort."}\NormalTok{)}
  \KeywordTok{select}\NormalTok{ \{}
  \KeywordTok{case}\NormalTok{ <-time.After(}\DecValTok{10}\NormalTok{ * time.Second):}
    \CommentTok{// Do nothing.}
  \KeywordTok{case}\NormalTok{ <-abort:}
\NormalTok{    fmt.Println(}\StringTok{"Launch aborted!"}\NormalTok{)}
    \KeywordTok{return} 
\NormalTok{    \}}
\NormalTok{  launch()}
\NormalTok{\}}
\end{Highlighting}
\end{Shaded}

\hypertarget{ux6ce8ux610fux4e8bux9879}{%
\subsection{注意事项}\label{ux6ce8ux610fux4e8bux9879}}

\hypertarget{ux591aux4e2a-chan-ux540cux65f6ux6709ux6570ux636eux7684ux65f6ux5019ux600eux4e48ux529e}{%
\subsubsection{多个 chan
同时有数据的时候怎么办?}\label{ux591aux4e2a-chan-ux540cux65f6ux6709ux6570ux636eux7684ux65f6ux5019ux600eux4e48ux529e}}

随机选择一个执行

\hypertarget{goroutineux6cc4ux9732}{%
\subsubsection{goroutine泄露}\label{goroutineux6cc4ux9732}}

\begin{Shaded}
\begin{Highlighting}[]
\KeywordTok{func}\NormalTok{ main() \{}
  \CommentTok{// ...create abort channel...}
\NormalTok{  fmt.Println(}\StringTok{"Commencing countdown.  Press return to abort."}\NormalTok{)}
\NormalTok{  tick := time.Tick(}\DecValTok{1}\NormalTok{ * time.Second)}\CommentTok{// 行为类似创建一个goroutine不断的向tick发送事件}
  \KeywordTok{for}\NormalTok{ countdown := }\DecValTok{10}\NormalTok{; countdown > }\DecValTok{0}\NormalTok{; countdown-- \{}
\NormalTok{    fmt.Println(countdown)}
    \KeywordTok{select}\NormalTok{ \{}
    \KeywordTok{case}\NormalTok{ <-tick:}
      \CommentTok{// Do nothing.}
    \KeywordTok{case}\NormalTok{ <-abort:}
\NormalTok{      fmt.Println(}\StringTok{"Launch aborted!"}\NormalTok{)}
      \KeywordTok{return} \CommentTok{// 返回之后,tick没有对象去取事件,但是却有一个goroutine再不断的发事件给tick.所以goroutine泄露}
\NormalTok{    \}}
\NormalTok{  \}}
\NormalTok{  launch()}
\NormalTok{\}}
\end{Highlighting}
\end{Shaded}

记得关闭它

\begin{Shaded}
\begin{Highlighting}[]
\NormalTok{ticker := time.NewTicker(}\DecValTok{1}\NormalTok{ * time.Second)}
\NormalTok{<-ticker.C }\CommentTok{// receive from the ticker's channel}
\NormalTok{ticker.Stop() }\CommentTok{// cause the ticker's goroutine to terminate (}\AlertTok{FIXME}\CommentTok{)}
\end{Highlighting}
\end{Shaded}

\hypertarget{ux975eux963bux585eux4fe1ux9053}{%
\subsubsection{非阻塞信道}\label{ux975eux963bux585eux4fe1ux9053}}

\begin{Shaded}
\begin{Highlighting}[]
\KeywordTok{select}\NormalTok{ \{}
\KeywordTok{case}\NormalTok{ <-abort:}
\NormalTok{  fmt.Printf(}\StringTok{"Launch aborted!}\CharTok{\textbackslash{}n}\StringTok{"}\NormalTok{)}
  \KeywordTok{return}
\KeywordTok{default}\NormalTok{:}
  \CommentTok{// do nothing}
\NormalTok{\}}
\end{Highlighting}
\end{Shaded}

\hypertarget{nil-chan}{%
\subsubsection{nil chan}\label{nil-chan}}

往nil chan发送或者接受数据都是永久阻塞。

\begin{Shaded}
\begin{Highlighting}[]
\OtherTok{nil}\NormalTok{ <- }\StringTok{"some"} \CommentTok{// 发送}
\NormalTok{<- }\OtherTok{nil}        \CommentTok{// 接受}
\end{Highlighting}
\end{Shaded}

\hypertarget{ux5e76ux53d1ux76eeux5f55ux904dux5386}{%
\section{并发目录遍历}\label{ux5e76ux53d1ux76eeux5f55ux904dux5386}}

\hypertarget{ux7b2cux4e00ux7248-1}{%
\subsection{第一版}\label{ux7b2cux4e00ux7248-1}}

\begin{Shaded}
\begin{Highlighting}[]
\CommentTok{// walkDir recursively walks the file tree rooted at dir}
\CommentTok{// and sends the size of each found file on fileSizes.}
\KeywordTok{func}\NormalTok{ walkDir(dir }\DataTypeTok{string}\NormalTok{, fileSizes }\KeywordTok{chan}\NormalTok{<- }\DataTypeTok{int64}\NormalTok{) \{}
  \KeywordTok{for}\NormalTok{ _, entry := }\KeywordTok{range}\NormalTok{ dirents(dir) \{}
    \KeywordTok{if}\NormalTok{ entry.IsDir() \{}
\NormalTok{      subdir := filepath.Join(dir, entry.Name())}
\NormalTok{      walkDir(subdir, fileSizes)}
\NormalTok{    \}}\KeywordTok{else}\NormalTok{ \{}
\NormalTok{      fileSizes <- entry.Size()}
\NormalTok{    \}}
\NormalTok{  \}}
\NormalTok{\}}
\CommentTok{// dirents returns the entries of directory dir.}
\KeywordTok{func}\NormalTok{ dirents(dir }\DataTypeTok{string}\NormalTok{) []os.FileInfo \{}
\NormalTok{  entries, err := ioutil.ReadDir(dir)}
  \KeywordTok{if}\NormalTok{ err != }\OtherTok{nil}\NormalTok{ \{}
\NormalTok{    fmt.Fprintf(os.Stderr, }\StringTok{"du1: %v}\CharTok{\textbackslash{}n}\StringTok{"}\NormalTok{, err)}
    \KeywordTok{return} \OtherTok{nil}
\NormalTok{  \}}
  \KeywordTok{return}\NormalTok{ entries}
\NormalTok{\}}
\end{Highlighting}
\end{Shaded}

\begin{Shaded}
\begin{Highlighting}[]
\KeywordTok{func}\NormalTok{ main() \{}
  \CommentTok{// Determine the initial directories.}
\NormalTok{  flag.Parse()}
\NormalTok{  roots := flag.Args()}
  \KeywordTok{if} \BuiltInTok{len}\NormalTok{(roots) == }\DecValTok{0}\NormalTok{ \{}
\NormalTok{    roots = []}\DataTypeTok{string}\NormalTok{\{}\StringTok{"."}\NormalTok{\}}
\NormalTok{  \}}
  \CommentTok{// Traverse the file tree.}
\NormalTok{  fileSizes := }\BuiltInTok{make}\NormalTok{(}\KeywordTok{chan} \DataTypeTok{int64}\NormalTok{)}
  \KeywordTok{go} \KeywordTok{func}\NormalTok{() \{}
    \KeywordTok{for}\NormalTok{ _, root := }\KeywordTok{range}\NormalTok{ roots \{}
\NormalTok{      walkDir(root, fileSizes)}
\NormalTok{    \}}
    \BuiltInTok{close}\NormalTok{(fileSizes)}
\NormalTok{  \}()}
  \CommentTok{// Print the results.}
  \KeywordTok{var}\NormalTok{ nfiles, nbytes }\DataTypeTok{int64}
  \KeywordTok{for}\NormalTok{ size := }\KeywordTok{range}\NormalTok{ fileSizes \{}
\NormalTok{    nfiles++}
\NormalTok{    nbytes += size}
\NormalTok{  \}}
\NormalTok{  printDiskUsage(nfiles, nbytes)}
\NormalTok{\}}

\KeywordTok{func}\NormalTok{ printDiskUsage(nfiles, nbytes }\DataTypeTok{int64}\NormalTok{) \{}
\NormalTok{  fmt.Printf(}\StringTok{"%d files  %.1f GB}\CharTok{\textbackslash{}n}\StringTok{"}\NormalTok{, nfiles, }\DataTypeTok{float64}\NormalTok{(nbytes)/}\FloatTok{1e9}\NormalTok{)}
\NormalTok{\}}
\end{Highlighting}
\end{Shaded}

\hypertarget{ux6539ux8fdbux7248}{%
\subsection{改进版}\label{ux6539ux8fdbux7248}}

\hypertarget{ux6bcfux9694ux4e00ux6bb5ux65f6ux95f4ux8f93ux51faux4fe1ux606f}{%
\subsubsection{每隔一段时间输出信息}\label{ux6bcfux9694ux4e00ux6bb5ux65f6ux95f4ux8f93ux51faux4fe1ux606f}}

参数v控制是否输出filesize,每隔一段时间输出

\begin{Shaded}
\begin{Highlighting}[]
\KeywordTok{var}\NormalTok{ verbose = flag.Bool(}\StringTok{"v"}\NormalTok{, }\OtherTok{false}\NormalTok{, }\StringTok{"show verbose progress messages"}\NormalTok{)}
\KeywordTok{func}\NormalTok{ main() \{}
  \CommentTok{// ...start background goroutine...}
  \CommentTok{// Print the results periodically.}
  \KeywordTok{var}\NormalTok{ tick <-}\KeywordTok{chan}\NormalTok{ time.Time}
  \KeywordTok{if}\NormalTok{ *verbose \{}
\NormalTok{    tick = time.Tick(}\DecValTok{500}\NormalTok{ * time.Millisecond)}
\NormalTok{  \}}
  \KeywordTok{var}\NormalTok{ nfiles, nbytes }\DataTypeTok{int64}
\NormalTok{loop:}
  \KeywordTok{for}\NormalTok{ \{}
    \KeywordTok{select}\NormalTok{ \{}
    \KeywordTok{case}\NormalTok{ size, ok := <-fileSizes:}
      \KeywordTok{if}\NormalTok{ !ok \{}
        \KeywordTok{break}\NormalTok{ loop }\CommentTok{// fileSizes was closed}
\NormalTok{      \}}
\NormalTok{      nfiles++}
\NormalTok{      nbytes += size}
    \KeywordTok{case}\NormalTok{ <-tick: }\CommentTok{// 有么有goroutine泄露?}
\NormalTok{      printDiskUsage(nfiles, nbytes)}
\NormalTok{    \}}
\NormalTok{  \}}
\NormalTok{  printDiskUsage(nfiles, nbytes) }\CommentTok{// final totals}
\NormalTok{\}}
\end{Highlighting}
\end{Shaded}

\hypertarget{ux5145ux5206ux5229ux7528ux597dux5e76ux53d1}{%
\subsubsection{充分利用好并发}\label{ux5145ux5206ux5229ux7528ux597dux5e76ux53d1}}

\begin{Shaded}
\begin{Highlighting}[]
\KeywordTok{func}\NormalTok{ main() \{}
  \CommentTok{// ...determine roots...}
  \CommentTok{// Traverse each root of the file tree in parallel.}
\NormalTok{  fileSizes := }\BuiltInTok{make}\NormalTok{(}\KeywordTok{chan} \DataTypeTok{int64}\NormalTok{)}
  \KeywordTok{var}\NormalTok{ n sync.WaitGroup}
  \KeywordTok{for}\NormalTok{ _, root := }\KeywordTok{range}\NormalTok{ roots \{}
\NormalTok{    n.Add(}\DecValTok{1}\NormalTok{)}
    \KeywordTok{go}\NormalTok{ walkDir(root, &n, fileSizes)}
\NormalTok{  \}}
  \KeywordTok{go} \KeywordTok{func}\NormalTok{() \{}
\NormalTok{    n.Wait()}
    \BuiltInTok{close}\NormalTok{(fileSizes)}
\NormalTok{  \}()}
  \CommentTok{// ...select loop... 上一个例子的信息输出循环}
\NormalTok{\}}

\KeywordTok{func}\NormalTok{ walkDir(dir }\DataTypeTok{string}\NormalTok{, n *sync.WaitGroup, fileSizes }\KeywordTok{chan}\NormalTok{<- }\DataTypeTok{int64}\NormalTok{) \{}
  \KeywordTok{defer}\NormalTok{ n.Done()}
  \KeywordTok{for}\NormalTok{ _, entry := }\KeywordTok{range}\NormalTok{ dirents(dir) \{}
    \KeywordTok{if}\NormalTok{ entry.IsDir() \{}
\NormalTok{      n.Add(}\DecValTok{1}\NormalTok{)}
\NormalTok{      subdir := filepath.Join(dir, entry.Name())}
      \KeywordTok{go}\NormalTok{ walkDir(subdir, n, fileSizes)}
\NormalTok{    \} }\KeywordTok{else}\NormalTok{ \{}
\NormalTok{      fileSizes <- entry.Size()}
\NormalTok{    \}}
\NormalTok{  \}}
\NormalTok{\}}
\end{Highlighting}
\end{Shaded}

\hypertarget{ux9650ux5236ux4e0aux4e00ux4e2aux4f8bux5b50ux7684ux5e76ux53d1ux91cf}{%
\subsubsection{限制上一个例子的并发量}\label{ux9650ux5236ux4e0aux4e00ux4e2aux4f8bux5b50ux7684ux5e76ux53d1ux91cf}}

\begin{Shaded}
\begin{Highlighting}[]
\CommentTok{// sema is a counting semaphore for limiting concurrency in dirents.}
\KeywordTok{var}\NormalTok{ sema = }\BuiltInTok{make}\NormalTok{(}\KeywordTok{chan} \KeywordTok{struct}\NormalTok{\{\}, }\DecValTok{20}\NormalTok{)}
\CommentTok{// dirents returns the entries of directory dir.}
\KeywordTok{func}\NormalTok{ dirents(dir }\DataTypeTok{string}\NormalTok{) []os.FileInfo \{}
\NormalTok{  sema <- }\KeywordTok{struct}\NormalTok{\{\}\{\}        }\CommentTok{// acquire token}
  \KeywordTok{defer} \KeywordTok{func}\NormalTok{() \{ <-sema \}() }\CommentTok{// release token}
  \CommentTok{// ...}
\NormalTok{\}}
\end{Highlighting}
\end{Shaded}

\hypertarget{ux53d6ux6d88}{%
\section{取消}\label{ux53d6ux6d88}}

\begin{Shaded}
\begin{Highlighting}[]
\KeywordTok{var}\NormalTok{ done = }\BuiltInTok{make}\NormalTok{(}\KeywordTok{chan} \KeywordTok{struct}\NormalTok{\{\})}
\KeywordTok{func}\NormalTok{ cancelled() }\DataTypeTok{bool}\NormalTok{ \{}
  \KeywordTok{select}\NormalTok{ \{}
  \KeywordTok{case}\NormalTok{ <-done:}
    \KeywordTok{return} \OtherTok{true}
  \KeywordTok{default}\NormalTok{:}
    \KeywordTok{return} \OtherTok{false}
\NormalTok{  \}}
\NormalTok{\}}
\end{Highlighting}
\end{Shaded}

\begin{Shaded}
\begin{Highlighting}[]
\CommentTok{// Cancel traversal when input is detected.}
\KeywordTok{go} \KeywordTok{func}\NormalTok{() \{}
\NormalTok{  os.Stdin.Read(}\BuiltInTok{make}\NormalTok{([]}\DataTypeTok{byte}\NormalTok{, }\DecValTok{1}\NormalTok{)) }\CommentTok{// read a single byte}
  \BuiltInTok{close}\NormalTok{(done)}
\NormalTok{\}()}
\end{Highlighting}
\end{Shaded}

\begin{Shaded}
\begin{Highlighting}[]
\KeywordTok{for}\NormalTok{ \{}
  \KeywordTok{select}\NormalTok{ \{}
  \KeywordTok{case}\NormalTok{ <-done: }\CommentTok{// 为什么多个地方有这个语句? 因为在一个关闭的通道上取东西,直接返回}
    \CommentTok{// Drain fileSizes to allow existing goroutines to finish.}
    \KeywordTok{for} \KeywordTok{range}\NormalTok{ fileSizes \{}
      \CommentTok{// Do nothing.}
\NormalTok{    \}}
    \KeywordTok{return}
  \KeywordTok{case}\NormalTok{ size, ok := <-fileSizes:}
    \CommentTok{// ...}
\NormalTok{  \}}
\NormalTok{\}}
\end{Highlighting}
\end{Shaded}

goroutine各自分别退出自己。

\begin{Shaded}
\begin{Highlighting}[]
\KeywordTok{func}\NormalTok{ walkDir(dir }\DataTypeTok{string}\NormalTok{, n *sync.WaitGroup, fileSizes }\KeywordTok{chan}\NormalTok{<- }\DataTypeTok{int64}\NormalTok{) \{}
  \KeywordTok{defer}\NormalTok{ n.Done()}
  \KeywordTok{if}\NormalTok{ cancelled() \{}
    \KeywordTok{return}
\NormalTok{  \}}
  \KeywordTok{for}\NormalTok{ _, entry := }\KeywordTok{range}\NormalTok{ dirents(dir) \{}
    \CommentTok{// ...}
\NormalTok{  \}}
\NormalTok{\}}
\KeywordTok{func}\NormalTok{ dirents(dir }\DataTypeTok{string}\NormalTok{) []os.FileInfo \{}
  \KeywordTok{select}\NormalTok{ \{}
  \KeywordTok{case}\NormalTok{ sema <- }\KeywordTok{struct}\NormalTok{\{\}\{\}: }\CommentTok{// acquire token}
  \KeywordTok{case}\NormalTok{ <-done:}
    \KeywordTok{return} \OtherTok{nil} \CommentTok{// cancelled}
\NormalTok{  \}}
  \KeywordTok{defer} \KeywordTok{func}\NormalTok{() \{ <-sema \}() }\CommentTok{// release token}
  \CommentTok{// ...read directory...}
\NormalTok{\}}
\end{Highlighting}
\end{Shaded}

\hypertarget{ux91cdux8981ux7684ux6d4bux8bd5ux6280ux5de7}{%
\subsection{重要的测试技巧}\label{ux91cdux8981ux7684ux6d4bux8bd5ux6280ux5de7}}

Q :
其实,你自己是很难判断是否把所有的goroutine都关闭并推出了。那么如何在main函数返回的时候,确定还有没有goroutine再运行呢?

A :
在main函数退出之前,调用一个panic,看运行时堆栈信息。如果信息只有main函数的堆栈则说明所有goroutine都退出了。否则,就是说有的goroutine没有退出。

\hypertarget{ux804aux5929ux670dux52a1ux5668}{%
\section{聊天服务器}\label{ux804aux5929ux670dux52a1ux5668}}

\begin{Shaded}
\begin{Highlighting}[]
\KeywordTok{func}\NormalTok{ main() \{}
\NormalTok{  listener, err := net.Listen(}\StringTok{"tcp"}\NormalTok{, }\StringTok{"localhost:8000"}\NormalTok{)}
  \KeywordTok{if}\NormalTok{ err != }\OtherTok{nil}\NormalTok{ \{}
\NormalTok{    log.Fatal(err)}
\NormalTok{  \}}
  \KeywordTok{go}\NormalTok{ broadcaster() }\CommentTok{// 广播消息}
  \KeywordTok{for}\NormalTok{ \{}
\NormalTok{    conn, err := listener.Accept()}
    \KeywordTok{if}\NormalTok{ err != }\OtherTok{nil}\NormalTok{ \{}
\NormalTok{      log.Print(err)}
      \KeywordTok{continue} 
\NormalTok{    \}}
    \KeywordTok{go}\NormalTok{ handleConn(conn) }\CommentTok{// 管理客户端发来的连接}
\NormalTok{  \}}
\NormalTok{\}}
\end{Highlighting}
\end{Shaded}

\begin{Shaded}
\begin{Highlighting}[]
\KeywordTok{type}\NormalTok{ client }\KeywordTok{chan}\NormalTok{<- }\DataTypeTok{string} \CommentTok{// an outgoing message channel}
\KeywordTok{var}\NormalTok{ (}
\NormalTok{  entering = }\BuiltInTok{make}\NormalTok{(}\KeywordTok{chan}\NormalTok{ client) }\CommentTok{// 这个entering chan 将输入输出什么? 应该是chan<- string}
\NormalTok{  leaving  = }\BuiltInTok{make}\NormalTok{(}\KeywordTok{chan}\NormalTok{ client)}
\NormalTok{  messages = }\BuiltInTok{make}\NormalTok{(}\KeywordTok{chan} \DataTypeTok{string}\NormalTok{) }\CommentTok{// all incoming client messages}
\NormalTok{)}
\KeywordTok{func}\NormalTok{ broadcaster() \{}
\NormalTok{  clients := }\BuiltInTok{make}\NormalTok{(}\KeywordTok{map}\NormalTok{[client]}\DataTypeTok{bool}\NormalTok{) }\CommentTok{// all connected clients (chan是可hash的么?)}
  \KeywordTok{for}\NormalTok{ \{}
    \KeywordTok{select}\NormalTok{ \{}
    \KeywordTok{case}\NormalTok{ msg := <-messages:}
      \CommentTok{// Broadcast incoming message to all}
      \CommentTok{// clients' outgoing message channels.}
      \KeywordTok{for}\NormalTok{ cli := }\KeywordTok{range}\NormalTok{ clients \{}
\NormalTok{        cli <- msg}
\NormalTok{      \}}
    \KeywordTok{case}\NormalTok{ cli := <-entering:}
\NormalTok{      clients[cli] = }\OtherTok{true}
    \KeywordTok{case}\NormalTok{ cli := <-leaving:}
      \BuiltInTok{delete}\NormalTok{(clients, cli)}
      \BuiltInTok{close}\NormalTok{(cli)}
\NormalTok{    \}}
\NormalTok{  \}}
\NormalTok{\}}
\end{Highlighting}
\end{Shaded}

\begin{Shaded}
\begin{Highlighting}[]
\KeywordTok{func}\NormalTok{ handleConn(conn net.Conn) \{}
\NormalTok{  ch := }\BuiltInTok{make}\NormalTok{(}\KeywordTok{chan} \DataTypeTok{string}\NormalTok{) }\CommentTok{// outgoing client messages}
  \KeywordTok{go}\NormalTok{ clientWriter(conn, ch)}
\NormalTok{  who := conn.RemoteAddr().String()}
\NormalTok{  ch <- }\StringTok{"You are "}\NormalTok{ + who}
\NormalTok{  messages <- who + }\StringTok{" has arrived"}
  \CommentTok{// 把ch本身放到entering里面,不是从ch中取出string放入entering.因为entering的接收类型为chan<- string}
\NormalTok{  entering <- ch }
\NormalTok{  input := bufio.NewScanner(conn)}
  \KeywordTok{for}\NormalTok{ input.Scan() \{}
\NormalTok{    messages <- who + }\StringTok{": "}\NormalTok{ + input.Text()}
\NormalTok{  \}}
  \CommentTok{// }\AlertTok{NOTE}\CommentTok{: ignoring potential errors from input.Err()}
\NormalTok{  leaving <- ch}
\NormalTok{  messages <- who + }\StringTok{" has left"}
\NormalTok{  conn.Close()}
\NormalTok{\}}
\KeywordTok{func}\NormalTok{ clientWriter(conn net.Conn, ch <-}\KeywordTok{chan} \DataTypeTok{string}\NormalTok{) \{}
  \KeywordTok{for}\NormalTok{ msg := }\KeywordTok{range}\NormalTok{ ch \{ }\CommentTok{// 会泄露么?}
\NormalTok{    fmt.Fprintln(conn, msg) }\CommentTok{// }\AlertTok{NOTE}\CommentTok{: ignoring network errors}
\NormalTok{  \}}
\NormalTok{\}}
\end{Highlighting}
\end{Shaded}

