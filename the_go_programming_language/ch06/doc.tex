\hypertarget{ux65b9ux6cd5}{%
\section{方法}\label{ux65b9ux6cd5}}

\hypertarget{ux65b9ux6cd5ux58f0ux660e}{%
\subsection{6.1 方法声明}\label{ux65b9ux6cd5ux58f0ux660e}}

\hypertarget{ux65b9ux6cd5ux7684ux58f0ux660eux548cux666eux901aux51fdux6570ux7684ux58f0ux660eux7c7bux4f3cux53eaux662fux5728ux51fdux6570ux540dux5b57ux524dux9762ux591aux4e86ux4e00ux4e2aux53c2ux6570}{%
\subsubsection{方法的声明和普通函数的声明类似,只是在函数名字前面多了一个参数}\label{ux65b9ux6cd5ux7684ux58f0ux660eux548cux666eux901aux51fdux6570ux7684ux58f0ux660eux7c7bux4f3cux53eaux662fux5728ux51fdux6570ux540dux5b57ux524dux9762ux591aux4e86ux4e00ux4e2aux53c2ux6570}}

\begin{verbatim}
package geometry

import "math"

type Point struct{ X, Y float64 }

// traditional function
func Distance(p, q Point) float64 {
    return math.Hypot(q.X-p.X, q.Y-p.Y)
}

// same thing, but as a method of the Point type
func (p Point) Distance(q Point) float64 {
    return math.Hypot(q.X-p.X, q.Y-p.Y)
}
\end{verbatim}

\begin{itemize}
\tightlist
\item
  附加的参数p 称为方法的接收者
\item
  Go 语言中,接收者不使用特殊名(比如this 或者self
  );而是我们自己选择接收者名字
\item
  调用方法的时候,接收者在方法名的前面,这样就和声明保持一致
\end{itemize}

\begin{verbatim}
p := Point{1, 2}
q := Point{4, 6}
fmt.PrintIn(Distantce(p, q)) // "5", 函数调用
fmt.PrintIn(p.Distantce(q)) // "5", 函数调用
\end{verbatim}

\begin{itemize}
\tightlist
\item
  因为每一个类型有它自己的命名空间,所以我们能够在其他不同的类型中使用名字
  Distance 作为方法名
\end{itemize}

\begin{verbatim}
// A Path is a journey connecting the points with straight lines.
type Path []Point

// Distance returns the distance traveled along the path.
func (path Path) Distance() float64 {
    sum := 0.0
    for i := range path {
        if i > 0 {
            sum += path[i-1].Distance(path[i])
        }
    }
    return sum
}
\end{verbatim}

\begin{itemize}
\tightlist
\item
  Go
  和许多其他面向对象的语言不同,它可以将方法绑定到任何类型上。可以很方便地为简单的类型(如数字、字符串、slice
  、map
  ,甚至函数等)定义附加的行为。同一个包下的任何类型都可以声明方法,只要它的类型既不是指针类型也不是接口类型。
\end{itemize}

\begin{verbatim}
package main

import (
    "fmt"
)

type Greeting func(name string) string

func (p Greeting) Print(v string) {
    fmt.Println(v)
}

func display(name string) string {
    return name
}

func main() {
    var f Greeting

    f.Print("this is fine")
    //display.Print("This is fine") // 会报错
    f = display
    f.Print("this is fine")
}
\end{verbatim}

\begin{itemize}
\tightlist
\item
  使用方法的第一个好处:命名可以比函数更简短。在包的外部进行调用的时候,方法能够使用更加简短的名字且省略包的名字:
\end{itemize}

\begin{verbatim}
import "gopl.io/ch6/geometry"

perim := geometry.Path{{l, 1}, {5, 1}, {5, 4}, {1, 1}}
fmt.Println(geometry.PathDistance(perim)) //"12",独立函数
fmt.Println(perim.Distance()) //"12"geometry.Path 的方法
\end{verbatim}

\hypertarget{ux6307ux9488ux63a5ux6536ux8005ux7684ux65b9ux6cd5}{%
\subsection{6.2
指针接收者的方法}\label{ux6307ux9488ux63a5ux6536ux8005ux7684ux65b9ux6cd5}}

\hypertarget{ux7531ux4e8eux4e3bux8c03ux51fdux6570ux4f1aux590dux5236ux6bcfux4e00ux4e2aux5b9eux53c2ux53d8ux91cfux5982ux679cux51fdux6570ux9700ux8981ux66f4ux65b0ux4e00ux4e2aux53d8ux91cfux6216ux8005ux5982ux679cux4e00ux4e2aux5b9eux53c2ux592aux5927ux800cux6211ux4eecux5e0cux671bux907fux514dux590dux5236ux6574ux4e2aux5b9eux53c2ux56e0ux6b64ux6211ux4eecux5fc5ux987bux4f7fux7528ux6307ux9488ux6765ux4f20ux9012ux53d8ux91cfux7684ux5730ux5740}{%
\subsubsection{由于主调函数会复制每一个实参变量,如果函数需要更新一个变量,或者如果一个实参太大而我们希望避免复制整个实参,因此我们必须使用指针来传递变量的地址。}\label{ux7531ux4e8eux4e3bux8c03ux51fdux6570ux4f1aux590dux5236ux6bcfux4e00ux4e2aux5b9eux53c2ux53d8ux91cfux5982ux679cux51fdux6570ux9700ux8981ux66f4ux65b0ux4e00ux4e2aux53d8ux91cfux6216ux8005ux5982ux679cux4e00ux4e2aux5b9eux53c2ux592aux5927ux800cux6211ux4eecux5e0cux671bux907fux514dux590dux5236ux6574ux4e2aux5b9eux53c2ux56e0ux6b64ux6211ux4eecux5fc5ux987bux4f7fux7528ux6307ux9488ux6765ux4f20ux9012ux53d8ux91cfux7684ux5730ux5740}}

\begin{verbatim}
func (p *Point) ScaleBy(factor float64) {
    p.X *= factor
    p.Y *= factor
}
// 注:在真实的程序中,习惯上遵循如果Point 的任何一个方法使用指针接收者,那么所有的 Point 方法都应该使用指针接收者
\end{verbatim}

\begin{itemize}
\tightlist
\item
  命名类型(Point)与指向它们的指针(*Point)是唯一可以出现在接收者声明处的类型。而且,为防止混淆,不允许本身是指针的类型进行方法声明:
\end{itemize}

\begin{verbatim}
type P *int
func (P) f() { /* ... */ } // compile error: invalid receiver type
\end{verbatim}

\begin{itemize}
\tightlist
\item
  通过提供 *Point 能够调用 (*Point).ScaleBy 方法,比如:
\end{itemize}

\begin{verbatim}
r := &Point{1, 2}
r.ScaleBy(2)
fmt.Println(*r) // "{2, 4}"

// 或者:
p := Point{1, 2}
pptr := &p
pptr.ScaleBy(2)
fmt.Println(p) // "{2, 4}"

// 或者:
p := Point{1, 2}
(&p).ScaleBy(2)
fmt.Println(p) // "{2, 4}"

// 或者:
p.ScaleBy(2)

// 或者:
pptr.Distance(q)

// 错误:
Point{1, 2}.ScaleBy(2) // compile error: can't take address of Point
\end{verbatim}

\begin{itemize}
\tightlist
\item
  总结:三种合法情况

  \begin{enumerate}
  \def\labelenumi{\arabic{enumi}.}
  \tightlist
  \item
    实参和形参类型相同
  \item
    实参类型为 T 而形参为 *T
  \item
    实参类型为 *T 而形参为 T
  \end{enumerate}
\end{itemize}

\hypertarget{ux6307ux9488ux7c7bux578bux63a5ux53d7ux8005ux548cux975eux6307ux9488ux7c7bux578bux63a5ux53d7ux8005ux7684ux533aux522b}{%
\subsubsection{指针类型接受者和非指针类型接受者的区别}\label{ux6307ux9488ux7c7bux578bux63a5ux53d7ux8005ux548cux975eux6307ux9488ux7c7bux578bux63a5ux53d7ux8005ux7684ux533aux522b}}

\begin{verbatim}
package main

import "fmt"

type Mutatable struct {
    a int
    b int
}

func (m Mutatable) StayTheSame() { // m是传值传到StayTheSame中来的。所以这个函数中m是调用StayTheSame的变量的一个拷贝
    m.a = 5
    m.b = 7
}

func (m *Mutatable) Mutate() { // m 是调用Mutate的m的一个指针,这是一个巨大的区别
    m.a = 10
    m.b = 14
}

func main() {
    m := &Mutatable{0, 0}
    fmt.Println(m)
    m.StayTheSame()
    fmt.Println(m)
    m.Mutate()
    fmt.Println(m)

    fmt.Println("-----------")

    n := Mutatable{0, 0}
    fmt.Println(n)
    n.StayTheSame()
    fmt.Println(n)
    n.Mutate()
    fmt.Println(n)
}
\end{verbatim}

If you don't need to edit the receiver value, use a value receiver.Value
receivers are concurrency safe, while pointer receivers are not
concurrency safe.

\hypertarget{nil-ux662fux4e00ux4e2aux5408ux6cd5ux7684ux63a5ux6536ux8005}{%
\subsubsection{nil
是一个合法的接收者}\label{nil-ux662fux4e00ux4e2aux5408ux6cd5ux7684ux63a5ux6536ux8005}}

\begin{verbatim}

type IntList struct{
    Value int
    Tail *IntList
}

func (list *IntList) Sum() int{
    if (list == nil){
        return 0
    }

    return list.Value + list.Trail.Sum()
}
\end{verbatim}

示例:net/url包里 Values 类型定义的一部分

net/url

\begin{verbatim}
package url

// Values maps a string key to a list of values.
type Values map[string][]string
// Get returns the first value associated with the given key,
// or "" if there are none.
func (v Values) Get(key string) string {
     if vs := v[key]; len(vs) > 0 {
         return vs[0]
     }
     return ""
}
// Add adds the value to key.
// It appends to any existing values associated with key.
func (v Values) Add(key, value string) {
    v[key] = append(v[key], value)
}
\end{verbatim}

gopl.io/ch6/urlvalues

\begin{verbatim}
m := url.Values{"lang": {"en"}} // direct construction
m.Add("item", "1")
m.Add("item", "2")

fmt.Println(m.Get("lang")) // "en"
fmt.Println(m.Get("q"))    // ""
fmt.Println(m.Get("item")) // "1"      (first value)
fmt.Println(m["item"])     // "[1 2]"  (direct map access)

m = nil
fmt.Println(m.Get("item")) // ""
m.Add("item", "3")         // panic: assignment to entry in nil map
\end{verbatim}

\hypertarget{ux901aux8fc7ux7ed3ux6784ux4f53ux5185ux5d4cux7ec4ux6210ux7c7bux578b}{%
\subsection{6.3
通过结构体内嵌组成类型}\label{ux901aux8fc7ux7ed3ux6784ux4f53ux5185ux5d4cux7ec4ux6210ux7c7bux578b}}

gopl.io/ch6/coloredpoint

\begin{verbatim}
import "image/color"
type Point struct{ X, Y float64 }
type ColoredPoint struct {
    Point
    Color color.RGBA
}
\end{verbatim}

\begin{itemize}
\tightlist
\item
  内嵌可以使我们在定义 ColoredPoint 时得到一种句法上的简写形式
\end{itemize}

\begin{verbatim}
var cp ColoredPoint
cp.X = 1
fmt.Println(cp.Point.X) // "1"
cp.Point.Y = 2
fmt.Println(cp.Y) // "2"
\end{verbatim}

\begin{itemize}
\tightlist
\item
  我们能够通过类型为 ColoredPoint 的接收者调用内嵌类型 Point
  的方法,即使在 ColoredPoint 类型没有声明过这个方法:
\end{itemize}

\begin{verbatim}
red := color.RGBA{255, 0, 0, 255}
blue := color.RGBA{0, 0, 255, 255}
var p = ColoredPoint{Point{1, 1}, red}
var q = ColoredPoint{Point{5, 4}, blue}
fmt.Println(p.Distance(q.Point)) // "5"
p.ScaleBy(2)
q.ScaleBy(2)
fmt.Println(p.Distance(q.Point)) // "10"
\end{verbatim}

\begin{itemize}
\tightlist
\item
  注意: 这里 Point 类型并不能理解为是 ColoredPoint 类型的基类。Distance
  有一个形参 Point, q 不是 Point ,因此虽然 q 有一个内嵌的 Point
  字段,但是必须显式地使用它
\end{itemize}

\begin{verbatim}
p.Distance(q) // compile error: cannot use q (ColoredPoint) as Point
\end{verbatim}

\begin{itemize}
\tightlist
\item
  匿名字段类型可以是个指向命名类型的指针
\end{itemize}

\begin{verbatim}
type ColoredPoint struct {
    *Point
    Color color.RGBA
}

p := ColoredPoint{&Point{1, 1}, red}
q := ColoredPoint{&Point{5, 4}, blue}
fmt.Println(p.Distance(*q.Point)) // "5"
q.Point = p.Point                 // p and q now share the same Point
p.ScaleBy(2)
fmt.Println(*p.Point, *q.Point) // "{2 2} {2 2}"
\end{verbatim}

\begin{itemize}
\item
  当编译器处理选择子(比如
  p.ScaleBy)的时候,首先,它先查找到直接声明的方法 ScaleBy
  ,之后再从来自 ColoredPoint 的内嵌字段的方法中进行查找,再之后从Point
  和 RGBA 中内嵌字段的方法中进行查找,以此类推
\item
  方法只能在命名的类型(比如 Point )和指向它们的指针(*Point
  )中声明,但内嵌帮助我们能够在未命名的结构体类型中声明方法。
\end{itemize}

\begin{verbatim}
var (
    mu sync.Mutex // guards mapping 包级别变量
    mapping = make(map[string]string) // 包级别变量
)

func Lookup(key string) string {
    mu.Lock()
    v := mapping[key]
    mu.Unlock()
    return v
}

var cache = struct {
         sync.Mutex
         mapping map[string]string 
}{mapping: make(map[string]string),} // 匿名结构体变量(ps:struct 没有名字,只使用了一次)
         
func Lookup(key string) string {
         cache.Lock()
         v := cache.mapping[key]
         cache.Unlock()
         return v
}
\end{verbatim}

\hypertarget{ux65b9ux6cd5ux53d8ux91cfux4e0eux8868ux8fbeux5f0f}{%
\subsection{6.4
方法变量与表达式}\label{ux65b9ux6cd5ux53d8ux91cfux4e0eux8868ux8fbeux5f0f}}

\hypertarget{ux65b9ux6cd5ux53d8ux91cfux9009ux62e9ux5b50-p.distance-ux53efux4ee5ux8d4bux4e88ux4e00ux4e2aux65b9ux6cd5ux53d8ux91cfux5b83ux662fux4e00ux4e2aux51fdux6570ux628aux65b9ux6cd5-point.distance-ux7ed1ux5b9aux5230ux4e00ux4e2aux63a5ux6536ux8005-p-ux4e0a}{%
\subsubsection{方法变量:选择子 p.~Distance
可以赋予一个方法变量,它是一个函数,把方法( Point.Distance
)绑定到一个接收者 p
上}\label{ux65b9ux6cd5ux53d8ux91cfux9009ux62e9ux5b50-p.distance-ux53efux4ee5ux8d4bux4e88ux4e00ux4e2aux65b9ux6cd5ux53d8ux91cfux5b83ux662fux4e00ux4e2aux51fdux6570ux628aux65b9ux6cd5-point.distance-ux7ed1ux5b9aux5230ux4e00ux4e2aux63a5ux6536ux8005-p-ux4e0a}}

\begin{verbatim}
p := Point{1, 2}
q := Point{4, 6}

distanceFromP := p.Distance        // method value
fmt.Println(distanceFromP(q))      // "5"
var origin Point                   // {0, 0}
fmt.Println(distanceFromP(origin)) // "2.23606797749979", sqrt(5)

scaleP := p.ScaleBy // method value
scaleP(2)           // p becomes (2, 4)
scaleP(3)           //      then (6, 12)
scaleP(10)          //      then (60, 120)
\end{verbatim}

\hypertarget{ux65b9ux6cd5ux8868ux8fbeux5f0fux548cux8c03ux7528ux4e00ux4e2aux666eux901aux7684ux51fdux6570ux4e0dux540cux5728ux8c03ux7528ux65b9ux6cd5ux7684ux65f6ux5019ux5fc5ux987bux63d0ux4f9bux63a5ux6536ux8005ux5e76ux4e14ux6309ux7167ux9009ux62e9ux5b50ux7684ux8bedux6cd5ux8fdbux884cux8c03ux7528}{%
\subsubsection{方法表达式:和调用一个普通的函数不同,在调用方法的时候必须提供接收者,并且按照选择子的语法进行调用}\label{ux65b9ux6cd5ux8868ux8fbeux5f0fux548cux8c03ux7528ux4e00ux4e2aux666eux901aux7684ux51fdux6570ux4e0dux540cux5728ux8c03ux7528ux65b9ux6cd5ux7684ux65f6ux5019ux5fc5ux987bux63d0ux4f9bux63a5ux6536ux8005ux5e76ux4e14ux6309ux7167ux9009ux62e9ux5b50ux7684ux8bedux6cd5ux8fdbux884cux8c03ux7528}}

\begin{verbatim}
p := Point{1, 2}
q := Point{4, 6}

distance := Point.Distance   // method expression
fmt.Println(distance(p, q))  // "5"
fmt.Printf("%T\n", distance) // "func(Point, Point) float64"

scale := (*Point).ScaleBy
scale(&p, 2)
fmt.Println(p)            // "{2 4}"
fmt.Printf("%T\n", scale) // "func(*Point, float64)"
\end{verbatim}

\hypertarget{ux793aux4f8bux4f4dux5411ux91cf}{%
\subsection{6.5 示例:位向量}\label{ux793aux4f8bux4f4dux5411ux91cf}}

简单示例 \textbar{} 0000\ldots{}0000 \textbar{} 0000\ldots{}0000
\textbar{} \ldots{} \textbar{} 0000\ldots{}0000 \textbar{}

比如32号元素为真,则把上面从右向左的第32位二进制位置1. \#\# 6.6 封装
\#\#\# Go
语言只有一种方式控制命名的可见性:定义的时候,首字母大写的标识符是可以从包中导出的,而首字母没有大写的则不导出。

\begin{itemize}
\tightlist
\item
  结论1: 要封装一个对象,必须使用结构体。 示例种的 IntSet
  类型被声明为结构体但是它只有单个字段:
\end{itemize}

\begin{verbatim}
type IntSet struct {
    words []uint64
}
\end{verbatim}

而另一种定义则允许其他包内的使用方读取和改变这个 slice

\begin{verbatim}
type IntSet []uint64
\end{verbatim}

\begin{itemize}
\tightlist
\item
  结论2: 在Go 语言中封装的单元是包而不是类型.
\end{itemize}

\begin{verbatim}
package test

type Kkk struct {
    private string
    Public  string
}

package main

import (
    "./test"
    "fmt"
)

type some struct {
    words string
}

func main() {
    var fine = test.Kkk{}
    var good = some{}
    //fmt.Println(fine.private) // 不能访问其他包中结构体中的小写开头变量
    fmt.Println(fine.Public) // 可以访问其他包中结构体中的大写开头变量
    good.words = "long"
    fmt.Println(good.words) // 可以访问本包结构体中的小写开头变量
}
\end{verbatim}

\begin{itemize}
\tightlist
\item
  封装提供了三个优点

  \begin{enumerate}
  \def\labelenumi{\arabic{enumi}.}
  \tightlist
  \item
    因为使用方不能直接修改对象的变量,所以不需要更多的语句用来检查变量的值。
  \item
    隐藏实现细节可以防止使用方依赖的属性发生改变,使得设计者可以更加灵活地改变
    API 的实现而不破坏兼容性。
  \item
    防止使用者肆意地改变对象内的变量。
  \end{enumerate}
\end{itemize}

\begin{verbatim}
package log
type Logger struct {
    flags  int
    prefix string
    // ...
}
func (l *Logger) Flags() int
func (l *Logger) SetFlags(flag int)
func (l *Logger) Prefix() string
func (l *Logger) SetPrefix(prefix string)
\end{verbatim}

\begin{itemize}
\tightlist
\item
  封装并不总是必须的,比如
\end{itemize}

\begin{verbatim}
const day = 24 * time.Hour //day is a time.Duration
fmt.Println(day.Seconds()) // "86400"
\end{verbatim}
