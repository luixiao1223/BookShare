\hypertarget{ch01}{%
\subsection{ch01}\label{ch01}}

\hypertarget{hello-world}{%
\paragraph{hello world}\label{hello-world}}

\begin{verbatim}
package main   => 包
import "fmt"       => 引入包

func main() { 
    fmt.Println (” Hello ,世界”)
}
\end{verbatim}

运行

\$ go run helloworld.go

\$ go build helloworld.go

\hypertarget{ux547dux4ee4ux884cux53c2ux6570}{%
\subsubsection{命令行参数}\label{ux547dux4ee4ux884cux53c2ux6570}}

os.Args

for循环

for是 Go 里面的 \textbf{唯一循环语句}

\begin{verbatim}
for initialization; condition; post { 

}

for condition {
    
}

for {
    
}
\end{verbatim}

\hypertarget{ch02}{%
\subsection{ch02}\label{ch02}}

\hypertarget{ux540dux79f0}{%
\paragraph{名称}\label{ux540dux79f0}}

名称的开头是一个字母(Unicode中的字符即可)或下划线,后面可以跟任意数字和下划线,并区分大小写。

实体第1个字母的大小写决定其可见性是否跨包
如果名称以大写字母的开头,它是导出的,意味着它对包外是可见和可访问的,可以被自己包之外的其他程序所引用,像fmt包中的Printf包名本身总是由小写字母组成。

关键字:break default func inteface select case defer go map struct chan
else goto package switch const fallthrough if range type continue for
import return var

常量:true false iota nil

类型: int int8 int16 int32 int64 uint uint8 uint16 uint32 uint64
uintptr float32 float64 complex128 complex64 bool byte rune string error

函数: make len cap new append copy close delete complex real imag panic
recover

\hypertarget{ux58f0ux660e}{%
\paragraph{声明}\label{ux58f0ux660e}}

\hypertarget{ux53d8ux91cfvar}{%
\subparagraph{变量var}\label{ux53d8ux91cfvar}}

\begin{verbatim}
var name type = expression
var i = 3
var i int
var i int = 3
\end{verbatim}

\begin{itemize}
\tightlist
\item
  变量可以通过调用返回多个值的函数进行初始化:
\end{itemize}

\begin{verbatim}
var f, err = os.Open(name) 
\end{verbatim}

\begin{itemize}
\item
  短变量 name := expression //fixme
\item
  Go 不允许存在无用的临时变量,不然会出现编译错误
\end{itemize}

\begin{verbatim}
package main
import (
    "fmt"
    "os"
)
func main() {
    s, sep :=””,
    for_, arg :=range os.Args[l:] {
        s += sep +
            sep = ""
    }
    fmt.Pintln(s)
}
\end{verbatim}

\begin{verbatim}
package main

import (
    "fmt"
    "log"
    "os"
)

var cmd string = "string"

func main() {
    cmd, b := 1, 2 // 覆盖了包体外面的额string 类型的string, 此处的cmd编程了int类型
    a, b := "sring", 1
    a, c := 1, 2 //会报错,因为a是string类型,而不能复制为int类型。
    _, err := os.Getwd()
    if err != nil {
        log.Fatal("os.Getwe failed: %v", err)
    }
    fmt.Println("some", b, cmd, a)
}
\end{verbatim}

\hypertarget{ux5e38ux91cf-const}{%
\subparagraph{常量( const )}\label{ux5e38ux91cf-const}}

\begin{verbatim}
const boilingF=212.0
\end{verbatim}

\hypertarget{ux7c7bux578b-type}{%
\subparagraph{类型( type )}\label{ux7c7bux578b-type}}

type name underlying-type

\begin{verbatim}
type Celsius float64
\end{verbatim}

\hypertarget{ux51fdux6570-func}{%
\subparagraph{函数( func)}\label{ux51fdux6570-func}}

\begin{verbatim}
func (f Fahrenheit) String() string { 
    return fmt.Sprintf(” %g ”, f) 
}
\end{verbatim}

\hypertarget{ux6307ux9488}{%
\paragraph{指针}\label{ux6307ux9488}}

\begin{itemize}
\tightlist
\item
  函数返回局部变量的地址是非常安全的
\end{itemize}

\hypertarget{newux51fdux6570}{%
\paragraph{new函数}\label{newux51fdux6570}}

\begin{verbatim}
p := new(int) //*int 类型的 ,指向未命名的 int 变量
fmt.Println(p) // 输出" 0"
*p = 2 //把未命名的 
fmt.Println(*p) // 输出”2"
\end{verbatim}

\hypertarget{ux53d8ux91cfux751fux547dux5468ux671f}{%
\paragraph{变量生命周期}\label{ux53d8ux91cfux751fux547dux5468ux671f}}

\hypertarget{ux8d4bux503c}{%
\paragraph{赋值}\label{ux8d4bux503c}}

\begin{itemize}
\tightlist
\item
  多重赋值
\end{itemize}

\begin{verbatim}
x, y = y, x 
a[i], a[j] = a[j], a[i]
\end{verbatim}

\hypertarget{ux5305ux6587ux4ef6ux5bfcux5165}{%
\paragraph{包、文件、导入}\label{ux5305ux6587ux4ef6ux5bfcux5165}}

\begin{itemize}
\tightlist
\item
  导入一个没有被引用的包。会触发编译错误
\end{itemize}

\hypertarget{ux5305ux521dux59cbux5316}{%
\paragraph{包初始化}\label{ux5305ux521dux59cbux5316}}

\begin{verbatim}
func init() { /* ...*/}
\end{verbatim}

init 函数不能被调用和被引用,另一方面,它也是普通的函数
在每个文件里,当程序启动的时候,init函数按照它们声明的顺序自动执行。

\begin{verbatim}
package some

import "fmt"

var a = 1

func init() {
    a = a + 1
    fmt.Println(a)
}

func some() {
    fmt.Println("some")
}

func Print() {
    some()
    init() // some/some.go:18:2: undefined: init Error!!!
}
\end{verbatim}

\begin{verbatim}
package main

import (
    "./some"
)

func main() {
    some.Print()
}
\end{verbatim}

\hypertarget{ux4f5cux7528ux57df}{%
\paragraph{作用域}\label{ux4f5cux7528ux57df}}

\begin{verbatim}
if f, err := cs.Open (fname); err!= nil {// 编译错误 未使用
    return err 
}

f. Stat() // 编译错误:未定义
f. Close() // 编译错误:未定义
\end{verbatim}

\hypertarget{fallthrough}{%
\paragraph{fallthrough}\label{fallthrough}}

\begin{verbatim}
switch choice {
case "optionone":
    // some instructions 
    fallthrough // control will not come out from this case but will go to next case.
case "optiontwo":
   // some instructions 
default: 
   return 
}
\end{verbatim}
