\hypertarget{ux51fdux6570ux58f0ux660e}{%
\section{函数声明}\label{ux51fdux6570ux58f0ux660e}}

\begin{Shaded}
\begin{Highlighting}[]
\KeywordTok{func}\NormalTok{ name(parameter-list) (result-list) \{}
\NormalTok{  body}
\NormalTok{\}}
\end{Highlighting}
\end{Shaded}

\hypertarget{example}{%
\subsection{example}\label{example}}

\begin{Shaded}
\begin{Highlighting}[]
\KeywordTok{func}\NormalTok{ hypot(x,y }\DataTypeTok{float64}\NormalTok{) }\DataTypeTok{float64}\NormalTok{ \{}
  \KeywordTok{return}\NormalTok{ math.Sqrt(x*x + y*y)}
\NormalTok{\}}
\end{Highlighting}
\end{Shaded}

\hypertarget{ux540cux7c7bux578bux53efux4ee5ux653eux5728ux4e00ux8d77}{%
\subsection{同类型可以放在一起}\label{ux540cux7c7bux578bux53efux4ee5ux653eux5728ux4e00ux8d77}}

\begin{Shaded}
\begin{Highlighting}[]
\KeywordTok{func}\NormalTok{ f(i, j, k }\DataTypeTok{int}\NormalTok{, s, t }\DataTypeTok{string}\NormalTok{) \{}\CommentTok{/*...*/}\NormalTok{\}}
\KeywordTok{func}\NormalTok{ f(i }\DataTypeTok{int}\NormalTok{, j }\DataTypeTok{int}\NormalTok{, k }\DataTypeTok{int}\NormalTok{, s }\DataTypeTok{string}\NormalTok{, t }\DataTypeTok{string}\NormalTok{)\{}\CommentTok{/*...*/}\NormalTok{\}}
\end{Highlighting}
\end{Shaded}

\hypertarget{ux9700ux8981ux6ce8ux610fux70b9}{%
\subsection{需要注意点}\label{ux9700ux8981ux6ce8ux610fux70b9}}

\begin{enumerate}
\item
  go 语言没有默认参数的概念,也不能指定参数名

\begin{Shaded}
\begin{Highlighting}[]
\KeywordTok{def}\NormalTok{ some(name}\OperatorTok{=}\DecValTok{10}\NormalTok{, kkk}\OperatorTok{=}\DecValTok{20}\NormalTok{):}
    \BuiltInTok{print}\NormalTok{(name)}

\NormalTok{some(name}\OperatorTok{=}\DecValTok{11}\NormalTok{)}
\end{Highlighting}
\end{Shaded}
\item
  函数类型(具有相同形参列表和返回列表的函数是同种类型)
\item
  没有函数体的函数声明,可能是来自于go语言之外的语言实现的。

\begin{Shaded}
\begin{Highlighting}[]
\KeywordTok{package}\NormalTok{ math}
\KeywordTok{func}\NormalTok{ sin(x }\DataTypeTok{float64}\NormalTok{) }\DataTypeTok{float64} \CommentTok{// 使用汇编语言实现 (等到12章,才会明白)}
\end{Highlighting}
\end{Shaded}
\end{enumerate}

\hypertarget{ux9012ux5f52}{%
\section{递归}\label{ux9012ux5f52}}

\hypertarget{ux591aux503cux8fd4ux56de}{%
\section{多值返回}\label{ux591aux503cux8fd4ux56de}}

\hypertarget{ux5ffdux7565ux53c2ux6570}{%
\subsection{忽略参数}\label{ux5ffdux7565ux53c2ux6570}}

\begin{Shaded}
\begin{Highlighting}[]
\NormalTok{links, _ := findlines(url)}
\end{Highlighting}
\end{Shaded}

\hypertarget{ux4f20ux9012ux53c2ux6570}{%
\subsection{传递参数}\label{ux4f20ux9012ux53c2ux6570}}

\begin{Shaded}
\begin{Highlighting}[]
\KeywordTok{func}\NormalTok{ findLinksLog(url }\DataTypeTok{string}\NormalTok{) ([]}\DataTypeTok{string}\NormalTok{, }\DataTypeTok{error}\NormalTok{) \{}
\NormalTok{  log.Printf(}\StringTok{"findLinks %s"}\NormalTok{, url)}
  \KeywordTok{return}\NormalTok{ findLinks(url)}
\NormalTok{\}}
\end{Highlighting}
\end{Shaded}

\hypertarget{ux6709ux7684ux51fdux6570ux4e5fux63a5ux53d7ux591aux8fd4ux56deux503cux51fdux6570ux4f5cux4e3aux591aux53c2ux6570}{%
\subsection{有的函数也接受多返回值函数,作为多参数}\label{ux6709ux7684ux51fdux6570ux4e5fux63a5ux53d7ux591aux8fd4ux56deux503cux51fdux6570ux4f5cux4e3aux591aux53c2ux6570}}

\begin{Shaded}
\begin{Highlighting}[]
\NormalTok{log.Println(findLinks(url))}
\CommentTok{// 和下面的调用等价}
\NormalTok{links, err := findLinks(url)}
\NormalTok{log.Println(links, err)}
\end{Highlighting}
\end{Shaded}

\hypertarget{ux88f8ux8fd4ux56de}{%
\subsection{裸返回}\label{ux88f8ux8fd4ux56de}}

\begin{Shaded}
\begin{Highlighting}[]
\KeywordTok{func}\NormalTok{ CountWordsAndImages(url }\DataTypeTok{string}\NormalTok{) (words, images }\DataTypeTok{int}\NormalTok{, err }\DataTypeTok{error}\NormalTok{) \{}
\NormalTok{  resp, err := http.Get(url)}
  \KeywordTok{if}\NormalTok{ err != }\OtherTok{nil}\NormalTok{ \{}
    \KeywordTok{return}
\NormalTok{  \}}
\NormalTok{  doc, err := html.Parse(resp.Body)}
\NormalTok{  resp.Body.Close()}
  \KeywordTok{if}\NormalTok{ err != }\OtherTok{nil}\NormalTok{ \{}
\NormalTok{    err = fmt.Errorf(}\StringTok{"parsing HTML: %s"}\NormalTok{, err)}
    \KeywordTok{return}
\NormalTok{  \}}
  \CommentTok{//(这种匿名返回,会让人很难受,让然误解。所以尽量少用这种返回)}
\NormalTok{  words, images = countWordsAndImages(doc)}
  \KeywordTok{return}
\NormalTok{\}}

\KeywordTok{func}\NormalTok{ countWordsAndImages(n *html.Node) (words, images }\DataTypeTok{int}\NormalTok{) \{ }\CommentTok{/* ... */}\NormalTok{ \}}
\end{Highlighting}
\end{Shaded}

\hypertarget{ux9519ux8bef}{%
\section{错误}\label{ux9519ux8bef}}

这里主要讲错误的处理策略

\hypertarget{ux76f4ux63a5ux5411ux4e0aux4e00ux5c42ux8c03ux7528ux8005ux6c47ux62a5}{%
\subsection{直接向上一层调用者汇报}\label{ux76f4ux63a5ux5411ux4e0aux4e00ux5c42ux8c03ux7528ux8005ux6c47ux62a5}}

\begin{Shaded}
\begin{Highlighting}[]
\NormalTok{resp, err := http.Get(url)}
\KeywordTok{if}\NormalTok{ err != }\OtherTok{nil}\NormalTok{ \{}
  \KeywordTok{return} \OtherTok{nil}\NormalTok{, err}
\NormalTok{\}}
\end{Highlighting}
\end{Shaded}

\hypertarget{ux91cdux8bd5ux82e5ux5e72ux6b21ux518dux62a5ux9519ux9000ux51fa}{%
\subsection{重试若干次再报错退出}\label{ux91cdux8bd5ux82e5ux5e72ux6b21ux518dux62a5ux9519ux9000ux51fa}}

\begin{Shaded}
\begin{Highlighting}[]
\KeywordTok{func}\NormalTok{ WaitForServer(url }\DataTypeTok{string}\NormalTok{) }\DataTypeTok{error}\NormalTok{ \{}
  \KeywordTok{const}\NormalTok{ timeout = }\DecValTok{1}\NormalTok{ * time.Minute}
\NormalTok{  deadline := time.Now().Add(timeout)}
  \KeywordTok{for}\NormalTok{ tries := }\DecValTok{0}\NormalTok{; time.Now().Before(deadline); tries++ \{}
\NormalTok{    _, err := http.Head(url)}
    \KeywordTok{if}\NormalTok{ err == }\OtherTok{nil}\NormalTok{ \{}
      \KeywordTok{return} \OtherTok{nil} \CommentTok{// success}
\NormalTok{    \}}
\NormalTok{    log.Printf(}\StringTok{"server not responding (%s); retrying..."}\NormalTok{, err)}
\NormalTok{    time.Sleep(time.Second << }\DataTypeTok{uint}\NormalTok{(tries)) }\CommentTok{// exponential back-off}
\NormalTok{  \}}
  \KeywordTok{return}\NormalTok{ fmt.Errorf(}\StringTok{"server %s failed to respond after %s"}\NormalTok{, url, timeout)}
\NormalTok{\}}
\end{Highlighting}
\end{Shaded}

\hypertarget{ux76f4ux63a5ux7ec8ux6b62ux7a0bux5e8f}{%
\subsection{直接终止程序}\label{ux76f4ux63a5ux7ec8ux6b62ux7a0bux5e8f}}

\begin{Shaded}
\begin{Highlighting}[]
\CommentTok{// (In function main.)}
\KeywordTok{if}\NormalTok{ err := WaitForServer(url); err != }\OtherTok{nil}\NormalTok{ \{}
\NormalTok{  fmt.Fprintf(os.Stderr, }\StringTok{"Site is down: %v}\CharTok{\textbackslash{}n}\StringTok{"}\NormalTok{, err)}
\NormalTok{  os.Exit(}\DecValTok{1}\NormalTok{)}
\NormalTok{\}}
\end{Highlighting}
\end{Shaded}

应该由主程序来做。不应该由库函数来做。库函数应该报告错误就行了。

\begin{Shaded}
\begin{Highlighting}[]
\NormalTok{log.Fatalf }\CommentTok{// 可以实现日志输出}
\end{Highlighting}
\end{Shaded}

\hypertarget{ux67d0ux4e9bux60c5ux51b5ux4e0bux53eaux662fux8bb0ux5f55ux9519ux8befux4fe1ux606fux7136ux540eux7ee7ux7eedux8fd0ux884c}{%
\subsection{某些情况下,只是记录错误信息,然后继续运行}\label{ux67d0ux4e9bux60c5ux51b5ux4e0bux53eaux662fux8bb0ux5f55ux9519ux8befux4fe1ux606fux7136ux540eux7ee7ux7eedux8fd0ux884c}}

\begin{Shaded}
\begin{Highlighting}[]
\KeywordTok{if}\NormalTok{ err := WaitForServer(url); err != }\OtherTok{nil}\NormalTok{ \{}
\NormalTok{  log.Fatalf(}\StringTok{"Site is down: %v}\CharTok{\textbackslash{}n}\StringTok{"}\NormalTok{, err)}
\NormalTok{\}}
\end{Highlighting}
\end{Shaded}

\hypertarget{ux76f4ux63a5ux5ffdux7565ux6389ux9519ux8bef}{%
\subsection{直接忽略掉错误}\label{ux76f4ux63a5ux5ffdux7565ux6389ux9519ux8bef}}

\begin{Shaded}
\begin{Highlighting}[]
\NormalTok{dir, err := ioutil.TempDir(}\StringTok{""}\NormalTok{, }\StringTok{"scratch"}\NormalTok{)}
\KeywordTok{if}\NormalTok{ err != }\OtherTok{nil}\NormalTok{ \{}
  \KeywordTok{return}\NormalTok{ fmt.Errorf(}\StringTok{"failed to create temp dir: %v"}\NormalTok{, err)}
\NormalTok{\}}
\CommentTok{// ...use temp dir...}
\NormalTok{os.RemoveAll(dir) }\CommentTok{// 这个函数可能会错误,但是这里忽略了处理。}
\end{Highlighting}
\end{Shaded}

\hypertarget{ux51fdux6570ux53d8ux91cf}{%
\section{函数变量}\label{ux51fdux6570ux53d8ux91cf}}

\begin{Shaded}
\begin{Highlighting}[]
\KeywordTok{var}\NormalTok{ f }\KeywordTok{func}\NormalTok{(}\DataTypeTok{int}\NormalTok{) }\DataTypeTok{int}
\end{Highlighting}
\end{Shaded}

注意,函数变量之间不可以比较。所以不能把函数变量作为map的key值。但是函数类型可以和nil比较。

作为参数的函数变量

\begin{Shaded}
\begin{Highlighting}[]
\KeywordTok{func}\NormalTok{ forEachNode(n *html.Node, pre, post }\KeywordTok{func}\NormalTok{(n *html.Node) }\DataTypeTok{string}\NormalTok{)\{}
  \CommentTok{//body}
\NormalTok{\}}
\end{Highlighting}
\end{Shaded}

\hypertarget{ux533fux540dux51fdux6570}{%
\section{匿名函数}\label{ux533fux540dux51fdux6570}}

\begin{Shaded}
\begin{Highlighting}[]
\NormalTok{strings.}\KeywordTok{map}\NormalTok{(}\KeywordTok{func}\NormalTok{(r }\DataTypeTok{rune}\NormalTok{) }\DataTypeTok{rune}\NormalTok{ \{}\KeywordTok{return}\NormalTok{ r+}\DecValTok{1}\NormalTok{\}, }\StringTok{"HAL-9000"}\NormalTok{)}
\end{Highlighting}
\end{Shaded}

\hypertarget{ux95edux5305ux7684ux6982ux5ff5}{%
\subsection{闭包的概念}\label{ux95edux5305ux7684ux6982ux5ff5}}

\begin{Shaded}
\begin{Highlighting}[]
\KeywordTok{func}\NormalTok{ squares() }\KeywordTok{func}\NormalTok{() }\DataTypeTok{int}\NormalTok{ \{}
  \KeywordTok{var}\NormalTok{ x }\DataTypeTok{int}
  \KeywordTok{return} \KeywordTok{func}\NormalTok{() }\DataTypeTok{int}\NormalTok{ \{}
\NormalTok{    x++}
    \KeywordTok{return}\NormalTok{ x * x}
\NormalTok{  \}}
\NormalTok{\}}

\KeywordTok{func}\NormalTok{ main() \{}
\NormalTok{  f := squares()}
\NormalTok{  fmt.Println(f()) }\CommentTok{// "1"}
\NormalTok{  fmt.Println(f()) }\CommentTok{// "4"}
\NormalTok{  fmt.Println(f()) }\CommentTok{// "9"}
\NormalTok{  fmt.Println(f()) }\CommentTok{// "16"}
\NormalTok{  fb := squares()}
\NormalTok{  fmt.Println(fb()) }\CommentTok{// "1"}
\NormalTok{  fmt.Println(f())  }\CommentTok{// "25"}
\NormalTok{\}}
\end{Highlighting}
\end{Shaded}

PS :
(TonyHuiHUI)给出可以理解为,squares的调用后置,并没有清除squares的堆栈信息。所以x是可以被函数f访问到的。当第二次调用squares的时候再次创建了一个新的堆栈fb就可以访问到新的x了

\hypertarget{ux5bb9ux6613ux51faux9519ux7684ux5730ux65b9}{%
\subsection{容易出错的地方}\label{ux5bb9ux6613ux51faux9519ux7684ux5730ux65b9}}

\hypertarget{wrong}{%
\subsubsection{wrong}\label{wrong}}

\begin{Shaded}
\begin{Highlighting}[]
\KeywordTok{var}\NormalTok{ rmdirs []}\KeywordTok{func}\NormalTok{()}
\KeywordTok{for}\NormalTok{ _, dir := }\KeywordTok{range}\NormalTok{ tempDirs() \{}

\NormalTok{  os.MkdirAll(dir, }\DecValTok{0755}\NormalTok{)}
\NormalTok{  rmdirs = }\BuiltInTok{append}\NormalTok{(rmdirs, }\KeywordTok{func}\NormalTok{() \{}
\NormalTok{    os.RemoveAll(dir) }\CommentTok{// }\AlertTok{NOTE}\CommentTok{: incorrect!}
\NormalTok{  \})}
\NormalTok{\}}
\end{Highlighting}
\end{Shaded}

\hypertarget{right}{%
\subsubsection{right}\label{right}}

\begin{Shaded}
\begin{Highlighting}[]
\KeywordTok{var}\NormalTok{ rmdirs []}\KeywordTok{func}\NormalTok{()}
\KeywordTok{for}\NormalTok{ _, d := }\KeywordTok{range}\NormalTok{ tempDirs() \{}
\NormalTok{  dir := d               }\CommentTok{// }\AlertTok{NOTE}\CommentTok{: necessary!}
\NormalTok{  os.MkdirAll(dir, }\DecValTok{0755}\NormalTok{) }\CommentTok{// creates parent directories too}
\NormalTok{  rmdirs = }\BuiltInTok{append}\NormalTok{(rmdirs, }\KeywordTok{func}\NormalTok{() \{}
\NormalTok{    os.RemoveAll(dir)}
\NormalTok{  \})}
\NormalTok{\}}

\CommentTok{// ...do some work...}
\KeywordTok{for}\NormalTok{ _, rmdir := }\KeywordTok{range}\NormalTok{ rmdirs \{}
\NormalTok{  rmdir() }\CommentTok{// clean up}
\NormalTok{\}}
\end{Highlighting}
\end{Shaded}

\hypertarget{ux53d8ux957fux51fdux6570}{%
\section{变长函数}\label{ux53d8ux957fux51fdux6570}}

\begin{Shaded}
\begin{Highlighting}[]
\KeywordTok{func}\NormalTok{ sum(vals ...}\DataTypeTok{int}\NormalTok{) }\DataTypeTok{int}\NormalTok{ \{}
\NormalTok{  total := }\DecValTok{0}
  \KeywordTok{for}\NormalTok{ _, val := }\KeywordTok{range}\NormalTok{ vals \{}
\NormalTok{    total += val}
\NormalTok{  \}}
  \KeywordTok{return}\NormalTok{ total}
\NormalTok{\}}
\end{Highlighting}
\end{Shaded}

\hypertarget{ux7b49ux4ef7ux8c03ux7528}{%
\subsection{等价调用}\label{ux7b49ux4ef7ux8c03ux7528}}

\begin{Shaded}
\begin{Highlighting}[]
\NormalTok{fmt.Println(sum(}\DecValTok{1}\NormalTok{, }\DecValTok{2}\NormalTok{, }\DecValTok{3}\NormalTok{, }\DecValTok{4}\NormalTok{)) }\CommentTok{// "10"}
\CommentTok{// 等价的调用}
\NormalTok{values := []}\DataTypeTok{int}\NormalTok{\{}\DecValTok{1}\NormalTok{, }\DecValTok{2}\NormalTok{, }\DecValTok{3}\NormalTok{, }\DecValTok{4}\NormalTok{\}}
\NormalTok{fmt.Println(sum(values...)) }\CommentTok{// "10"}
\end{Highlighting}
\end{Shaded}

\hypertarget{ux4e0dux540cux7c7bux578b}{%
\subsection{不同类型}\label{ux4e0dux540cux7c7bux578b}}

\begin{Shaded}
\begin{Highlighting}[]
\KeywordTok{func}\NormalTok{ f(...}\DataTypeTok{int}\NormalTok{) \{\}}
\KeywordTok{func}\NormalTok{ g([]}\DataTypeTok{int}\NormalTok{) \{\}}
\end{Highlighting}
\end{Shaded}

\hypertarget{ux5ef6ux8fdfux51fdux6570}{%
\section{延迟函数}\label{ux5ef6ux8fdfux51fdux6570}}

\begin{Shaded}
\begin{Highlighting}[]
\KeywordTok{func}\NormalTok{ title(url }\DataTypeTok{string}\NormalTok{) }\DataTypeTok{error}\NormalTok{ \{}
\NormalTok{  resp, err := http.Get(url)}
  \KeywordTok{if}\NormalTok{ err != }\OtherTok{nil}\NormalTok{ \{}
    \KeywordTok{return}\NormalTok{ err}
\NormalTok{  \}}
  \CommentTok{// Check Content-Type is HTML (e.g., "text/html; charset=utf-8").}
\NormalTok{  ct := resp.Header.Get(}\StringTok{"Content-Type"}\NormalTok{)}
  \KeywordTok{if}\NormalTok{ ct != }\StringTok{"text/html"}\NormalTok{ && !strings.HasPrefix(ct, }\StringTok{"text/html;"}\NormalTok{) \{}
\NormalTok{    resp.Body.Close() }\CommentTok{// 调用了一次}
    \KeywordTok{return}\NormalTok{ fmt.Errorf(}\StringTok{"%s has type %s, not text/html"}\NormalTok{, url, ct)}
\NormalTok{  \}}
\NormalTok{  doc, err := html.Parse(resp.Body)}
\NormalTok{  resp.Body.Close() }\CommentTok{// 调用了一次}
  \KeywordTok{if}\NormalTok{ err != }\OtherTok{nil}\NormalTok{ \{}
    \KeywordTok{return}\NormalTok{ fmt.Errorf(}\StringTok{"parsing %s as HTML: %v"}\NormalTok{, url, err)}
\NormalTok{  \}}

\NormalTok{  visitNode := }\KeywordTok{func}\NormalTok{(n *html.Node) \{}
    \KeywordTok{if}\NormalTok{ n.Type == html.ElementNode && n.Data == }\StringTok{"title"}\NormalTok{ &&}
\NormalTok{      n.FirstChild != }\OtherTok{nil}\NormalTok{ \{}
\NormalTok{      fmt.Println(n.FirstChild.Data)}
\NormalTok{    \}}
\NormalTok{  \}}
\NormalTok{  forEachNode(doc, visitNode, }\OtherTok{nil}\NormalTok{)}
  \KeywordTok{return} \OtherTok{nil}
\NormalTok{\}}
\end{Highlighting}
\end{Shaded}

\hypertarget{defer}{%
\subsection{defer}\label{defer}}

\begin{Shaded}
\begin{Highlighting}[]
\KeywordTok{func}\NormalTok{ title(url }\DataTypeTok{string}\NormalTok{) }\DataTypeTok{error}\NormalTok{ \{}
\NormalTok{  resp, err := http.Get(url)}
  \KeywordTok{if}\NormalTok{ err != }\OtherTok{nil}\NormalTok{ \{}
    \KeywordTok{return}\NormalTok{ err}
\NormalTok{  \}}
  \KeywordTok{defer}\NormalTok{ resp.Body.Close() }\CommentTok{// 发生在return之后}
\NormalTok{  ct := resp.Header.Get(}\StringTok{"Content-Type"}\NormalTok{)}
  \KeywordTok{if}\NormalTok{ ct != }\StringTok{"text/html"}\NormalTok{ && !strings.HasPrefix(ct, }\StringTok{"text/html;"}\NormalTok{) \{}
    \KeywordTok{return}\NormalTok{ fmt.Errorf(}\StringTok{"%s has type %s, not text/html"}\NormalTok{, url, ct)}
\NormalTok{  \}}
\NormalTok{  doc, err := html.Parse(resp.Body)}
  \KeywordTok{if}\NormalTok{ err != }\OtherTok{nil}\NormalTok{ \{}
    \KeywordTok{return}\NormalTok{ fmt.Errorf(}\StringTok{"parsing %s as HTML: %v"}\NormalTok{, url, err)}
\NormalTok{  \}}
  \CommentTok{// ...print doc's title element...}
  \KeywordTok{return} \OtherTok{nil}
\NormalTok{\}}
\end{Highlighting}
\end{Shaded}

\hypertarget{ux6ce8ux610f}{%
\subsection{注意}\label{ux6ce8ux610f}}

\begin{enumerate}
\tightlist
\item
  defer 没有限制使用次数,执行的时候以调用defer的顺序倒序执行。
\item
  defer 语句的求值是在执行defer语句的时候执行。
\item
  defer 的执行在return语句之后。
\end{enumerate}

\hypertarget{ux6539ux53d8ux8fd4ux56deux503cux7ed3ux679c}{%
\subsection{改变返回值结果}\label{ux6539ux53d8ux8fd4ux56deux503cux7ed3ux679c}}

\begin{Shaded}
\begin{Highlighting}[]
\KeywordTok{func}\NormalTok{ double(x }\DataTypeTok{int}\NormalTok{) (result }\DataTypeTok{int}\NormalTok{) \{}
  \KeywordTok{defer} \KeywordTok{func}\NormalTok{() \{ fmt.Printf(}\StringTok{"double(%d) = %d}\CharTok{\textbackslash{}n}\StringTok{"}\NormalTok{, x, result) \}()}\CommentTok{// return 后执行打印操作}
  \KeywordTok{return}\NormalTok{ x + x}
\NormalTok{\}}
\NormalTok{_ = double(}\DecValTok{4}\NormalTok{)}
\CommentTok{// Output:}
\CommentTok{// "double(4) = 8"}
\end{Highlighting}
\end{Shaded}

\begin{Shaded}
\begin{Highlighting}[]
\KeywordTok{func}\NormalTok{ triple(x }\DataTypeTok{int}\NormalTok{) (result }\DataTypeTok{int}\NormalTok{) \{}
  \KeywordTok{defer} \KeywordTok{func}\NormalTok{() \{ result += x \}()}
  \KeywordTok{return}\NormalTok{ double(x)}
\NormalTok{\}}
\NormalTok{fmt.Println(triple(}\DecValTok{4}\NormalTok{)) }\CommentTok{// "12" 改变了返回值}
\end{Highlighting}
\end{Shaded}

\hypertarget{ux6587ux4ef6ux63cfux8ff0ux7b26ux5e94ux7528}{%
\subsection{文件描述符应用}\label{ux6587ux4ef6ux63cfux8ff0ux7b26ux5e94ux7528}}

\hypertarget{ux53efux80fdux4f1aux8017ux5c3dux6587ux4ef6ux63cfux8ff0ux7b26ux8d44ux6e90}{%
\subsubsection{可能会耗尽文件描述符资源}\label{ux53efux80fdux4f1aux8017ux5c3dux6587ux4ef6ux63cfux8ff0ux7b26ux8d44ux6e90}}

\begin{Shaded}
\begin{Highlighting}[]
\KeywordTok{for}\NormalTok{ _, filename := }\KeywordTok{range}\NormalTok{ filenames \{}
\NormalTok{  f, err := os.Open(filename)}
  \KeywordTok{if}\NormalTok{ err != }\OtherTok{nil}\NormalTok{ \{}
    \KeywordTok{return}\NormalTok{ err}
\NormalTok{  \}}
  \KeywordTok{defer}\NormalTok{ f.Close() }\CommentTok{// }\AlertTok{NOTE}\CommentTok{: risky; could run out of file descriptors}
  \CommentTok{// ...process f...}
\NormalTok{\}}
\end{Highlighting}
\end{Shaded}

\hypertarget{ux66f4ux597dux7684ux65b9ux6cd5}{%
\subsubsection{更好的方法}\label{ux66f4ux597dux7684ux65b9ux6cd5}}

\begin{Shaded}
\begin{Highlighting}[]
\KeywordTok{for}\NormalTok{ _, filename := }\KeywordTok{range}\NormalTok{ filenames \{}
  \KeywordTok{if}\NormalTok{ err := doFile(filename); err != }\OtherTok{nil}\NormalTok{ \{}
    \KeywordTok{return}\NormalTok{ err}
\NormalTok{  \}}
\NormalTok{\}}

\KeywordTok{func}\NormalTok{ doFile(filename }\DataTypeTok{string}\NormalTok{) }\DataTypeTok{error}\NormalTok{ \{}
\NormalTok{  f, err := os.Open(filename)}
  \KeywordTok{if}\NormalTok{ err != }\OtherTok{nil}\NormalTok{ \{}
    \KeywordTok{return}\NormalTok{ err}
\NormalTok{  \}}
  \KeywordTok{defer}\NormalTok{ f.Close()}
  \CommentTok{// ...process f...}
\NormalTok{\}}
\end{Highlighting}
\end{Shaded}

\hypertarget{ux5b95ux673apanic}{%
\section{宕机(panic)}\label{ux5b95ux673apanic}}

\hypertarget{ux6ce8ux610f-1}{%
\subsection{注意}\label{ux6ce8ux610f-1}}

\begin{enumerate}
\item
  宕机会导致程序退出,只有在十分严重的错误情况下才可以宕机。
\item
  当发生宕机时,所有的延迟函数以倒序执行,直到回到main函数

\begin{Shaded}
\begin{Highlighting}[]
\KeywordTok{func}\NormalTok{ main() \{}
\NormalTok{  f(}\DecValTok{3}\NormalTok{)}
\NormalTok{\}}

\KeywordTok{func}\NormalTok{ f(x }\DataTypeTok{int}\NormalTok{) \{}
\NormalTok{  fmt.Printf(}\StringTok{"f(%d)}\CharTok{\textbackslash{}n}\StringTok{"}\NormalTok{, x+}\DecValTok{0}\NormalTok{/x) }\CommentTok{// panics if x == 0}
  \KeywordTok{defer}\NormalTok{ fmt.Printf(}\StringTok{"defer %d}\CharTok{\textbackslash{}n}\StringTok{"}\NormalTok{, x)}
\NormalTok{  f(x - }\DecValTok{1}\NormalTok{)}
\NormalTok{\}}
\end{Highlighting}
\end{Shaded}

  outputs

\begin{Shaded}
\begin{Highlighting}[]
\ExtensionTok{f}\NormalTok{(3)}
\ExtensionTok{f}\NormalTok{(2)}
\ExtensionTok{f}\NormalTok{(1)}
\ExtensionTok{defer}\NormalTok{ 1}
\ExtensionTok{defer}\NormalTok{ 2}
\ExtensionTok{defer}\NormalTok{ 3}
\end{Highlighting}
\end{Shaded}
\end{enumerate}

\hypertarget{runtime}{%
\subsection{runtime}\label{runtime}}

runtime包提供了转储栈的方法使程序员可以诊断错误。

\begin{Shaded}
\begin{Highlighting}[]
\NormalTok{gopl.io/ch5/defer2}

\KeywordTok{func}\NormalTok{ main() \{}
  \KeywordTok{defer}\NormalTok{ printStack()}
\NormalTok{  f(}\DecValTok{3}\NormalTok{)}
\NormalTok{\}}

\KeywordTok{func}\NormalTok{ printStack() \{}
  \KeywordTok{var}\NormalTok{ buf [}\DecValTok{4096}\NormalTok{]}\DataTypeTok{byte}
\NormalTok{  n := runtime.Stack(buf[:], }\OtherTok{false}\NormalTok{)}
\NormalTok{  os.Stdout.Write(buf[:n])}
\NormalTok{\}}
\end{Highlighting}
\end{Shaded}

为什么可以打印出栈,因为go语言的宕机机制可以让延迟函数的执行在栈清理之前调用

\hypertarget{ux6062ux590d}{%
\section{恢复}\label{ux6062ux590d}}

recover可以劫持宕机,然后处理之后恢复运行

\begin{Shaded}
\begin{Highlighting}[]
\KeywordTok{func}\NormalTok{ Parse(input }\DataTypeTok{string}\NormalTok{) (s *Syntax, err }\DataTypeTok{error}\NormalTok{) \{}
  \KeywordTok{defer} \KeywordTok{func}\NormalTok{() \{}
    \KeywordTok{if}\NormalTok{ p := }\BuiltInTok{recover}\NormalTok{(); p != }\OtherTok{nil}\NormalTok{ \{}
\NormalTok{      err = fmt.Errorf(}\StringTok{"internal error: %v"}\NormalTok{, p) }\CommentTok{// 这里就会恢复,程序不会退出}
\NormalTok{    \}}
\NormalTok{  \}()}
  \CommentTok{// ...parser... 假如在这里发生宕机}
\NormalTok{\}}
\end{Highlighting}
\end{Shaded}

\begin{Shaded}
\begin{Highlighting}[]
\KeywordTok{func}\NormalTok{ soleTitle(doc *html.Node) (title }\DataTypeTok{string}\NormalTok{, err }\DataTypeTok{error}\NormalTok{) \{}
  \KeywordTok{type}\NormalTok{ bailout }\KeywordTok{struct}\NormalTok{\{\}}
  \KeywordTok{defer} \KeywordTok{func}\NormalTok{() \{}
    \KeywordTok{switch}\NormalTok{ p := }\BuiltInTok{recover}\NormalTok{(); p \{}
    \KeywordTok{case} \OtherTok{nil}\NormalTok{:}
      \CommentTok{// no panic}
    \KeywordTok{case}\NormalTok{ bailout\{\}:}
      \CommentTok{// "expected" panic}
\NormalTok{      err = fmt.Errorf(}\StringTok{"multiple title elements"}\NormalTok{)}
    \KeywordTok{default}\NormalTok{:}
      \BuiltInTok{panic}\NormalTok{(p) }\CommentTok{// unexpected panic; carry on panicking}
\NormalTok{    \}}
\NormalTok{  \}() }\CommentTok{// 定义了一个函数并调用之这是个匿名函数(延迟调用)}

  \CommentTok{// Bail out of recursion if we find more than one non-empty title.}
\NormalTok{  forEachNode(doc, }\KeywordTok{func}\NormalTok{(n *html.Node) \{}
    \KeywordTok{if}\NormalTok{ n.Type == html.ElementNode && n.Data == }\StringTok{"title"}\NormalTok{ &&}
\NormalTok{      n.FirstChild != }\OtherTok{nil}\NormalTok{ \{}
      \KeywordTok{if}\NormalTok{ title != }\StringTok{""}\NormalTok{ \{}
        \BuiltInTok{panic}\NormalTok{(bailout\{\}) }\CommentTok{// 宕机发生地}
\NormalTok{      \}}
\NormalTok{      title = n.FirstChild.Data}
\NormalTok{    \}}
\NormalTok{  \}, }\OtherTok{nil}\NormalTok{)}
  \KeywordTok{if}\NormalTok{ title == }\StringTok{""}\NormalTok{ \{}
    \KeywordTok{return} \StringTok{""}\NormalTok{, fmt.Errorf(}\StringTok{"no title element"}\NormalTok{)}
\NormalTok{  \}}
  \KeywordTok{return}\NormalTok{ title, }\OtherTok{nil}
\NormalTok{\}}
\end{Highlighting}
\end{Shaded}

