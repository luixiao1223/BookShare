\#反射

\begin{itemize}
\tightlist
\item
  Go
  语言提供了一种机制,在编译时不知道类型的情况下,可更新变量、在运行时查看值、调用方法以及直接对它们的布局进行操作,这种机制称为反射(reflection)
\end{itemize}

\hypertarget{ux4e3aux4ec0ux4e48ux4f7fux7528ux53cdux5c04}{%
\subsection{12.1
为什么使用反射}\label{ux4e3aux4ec0ux4e48ux4f7fux7528ux53cdux5c04}}

\begin{itemize}
\item
  有时我们需要写一个函数有能力统一处理各种值类型的函数,而这些类型可能无法共享同一个接口,也可能布局未知,也有可能这个类型在我们设计函数时还不存在
\item
  例子:fmt.Printf
  中的格式化逻辑,它可以输出任意类型的任意值,甚至是用户自定义的一个类型
\end{itemize}

\begin{verbatim}
func Sprint(x interface{}) string {
    type stringe interface{
        String() string
    }
    switch x := x.(type) {
    case stringer:
        return x.String()
    case string:
        return x
    case int:
        return strconv.Itoa(x)
    // 对 int16、uint32 等类型做类似处理
    case bool:
        if x{
            return "true"
        }
        return "false"
    default:
        // arry, chan, func, map, pointer, slice, struct
        return "???"
    }
}
\end{verbatim}

问题: 如何处理类似 {[}{]}float64、map{[}string{]}{[}{]}string
的其他类型、以及自己命名的类型,比如 url.Values? -
当我们无法透视一个未知类型的布局、代码无法继续时时,就需要反射了

\hypertarget{reflect.type-ux548c-reflect.value}{%
\subsection{12.2 reflect.Type 和
reflect.Value}\label{reflect.type-ux548c-reflect.value}}

\begin{itemize}
\tightlist
\item
  reflect 包定义了两个重要的类型:Type 和 Value
\end{itemize}

\hypertarget{reflect.type}{%
\subsubsection{reflect.Type}\label{reflect.type}}

\begin{itemize}
\tightlist
\item
  reflect.TypeOf 函数接受任何的 interface\{\}
  参数,并且把接口中的动态类型以 reflect.Type 形式返回。
\end{itemize}

\begin{verbatim}
t := reflect.TypeOf(3)  // 一个 reflect.Type
fmt.Println(t.String()) // "int"
fmt.Println(t)          // "int"
\end{verbatim}

\begin{itemize}
\tightlist
\item
  reflect.Type 满足 fmt.Stringer, fmt.Printf 提供了一个简写方式
  \%T,内部实现就使用了 reflect.TypeOf:
  \texttt{fmt.Printf("\%T\textbackslash{}n,\ e)\ //\ "int"}
\end{itemize}

\hypertarget{reflect.value}{%
\subsubsection{reflect.Value}\label{reflect.value}}

\begin{itemize}
\tightlist
\item
  reflect.ValueOf 函数接受任意的 interface\{\} 并将接口的动态值以
  reflect.Value 的形式返回
\end{itemize}

\begin{verbatim}
v := reflect.ValueOf(3)  // 一个reflect.Value
fmt.Println(v)           // "3"
fmt.Printf("%v\n", v)    // "3"
fmt.Println(v. String()) // 注意:"<int Value>"
\end{verbatim}

\begin{itemize}
\item
  reflect.Value 也满足 fmt.Stringer, fmt包的 \%v 功能会对 reflect.Value
  进行特殊处理
\item
  调用 Value 的 Type 方法会把它的类型以 reflect.Type 方式返回:
\end{itemize}

\begin{verbatim}
t := v.Type()           // 一个reflect.Type
fmt.Println(t.String()) // "int"
\end{verbatim}

\begin{itemize}
\tightlist
\item
  reflect.ValueOf 的逆操作是 reflect.Value.Interface
  方法。它返回一个interface\{\}接口值,与 reflect.Value
  包含同一个具体值。
\end{itemize}

\begin{verbatim}
v := reflect.ValueOf(3) // a reflect.Value
x := v.Interface()      // an interface{}
i := x.(int)            // an int x.(int)是一个类型断言
fmt.Printf("%d\n", i)   // "3"
\end{verbatim}

\begin{itemize}
\tightlist
\item
  格式化函数的第二次尝试
\end{itemize}

\begin{verbatim}
package format

import (
    "reflect"
    "strconv"
)

// Any formats any value as a string.
func Any(value interface{}) string {
    return formatAtom(reflect.ValueOf(value))
}

// formatAtom formats a value without inspecting its internal structure.
func formatAtom(v reflect.Value) string {
    switch v.Kind() {
    case reflect.Invalid:
        return "invalid"
    case reflect.Int, reflect.Int8, reflect.Int16,
        reflect.Int32, reflect.Int64:
        return strconv.FormatInt(v.Int(), 10)
    case reflect.Uint, reflect.Uint8, reflect.Uint16,
        reflect.Uint32, reflect.Uint64, reflect.Uintptr:
        return strconv.FormatUint(v.Uint(), 10)
    // ...floating-point and complex cases omitted for brevity...
    case reflect.Bool:
        return strconv.FormatBool(v.Bool())
    case reflect.String:
        return strconv.Quote(v.String())
    case reflect.Chan, reflect.Func, reflect.Ptr, reflect.Slice, reflect.Map:
        return v.Type().String() + " 0x" +
            strconv.FormatUint(uint64(v.Pointer()), 16)
    default: // reflect.Array, reflect.Struct, reflect.Interface
        return v.Type().String() + " value"
    }
}
\end{verbatim}

该函数把每个值当做一个没有内部结构且不可分割的物体。对于聚合类型(结构体和数组)以及接口,它只输出了值的类型;对于引用类型(通道、函数、指针、slice
和 map),它输出了类型和以十六进制表示的引用地址

\begin{verbatim}
    var x int64 = 1
    var d time.Duration = 1 * time.Nanosecond
    fmt.Println(format.Any(x))                  // "1"
    fmt.Println(format.Any(d))                  // "1"
    fmt.Println(format.Any([]int64{x}))         // "[]int64 0x8202b87b0"
    fmt.Println(format.Any([]time.Duration{d})) // "[]time.Duration 0x8202b87e0"
\end{verbatim}

\hypertarget{ux4f7fux7528-reflect.value-ux6765ux8bbeux7f6eux503c}{%
\subsection{12.5 使用 reflect.Value
来设置值}\label{ux4f7fux7528-reflect.value-ux6765ux8bbeux7f6eux503c}}

\begin{itemize}
\tightlist
\item
  reflect.Value 分为可寻址的和不可寻址的两类
\end{itemize}

\begin{verbatim}
x := 2                      // 值类型变量?
a := reflect.ValueOf(2)     // 2   int   no
b := reflect.ValueOf(x)     // 2   int   no
c := reflect.ValueOf(&x)    // &x *int   no
d := c.Elem()               // 2   int   yes (x)
\end{verbatim}

前三个都是保存的副本,不可寻址,所以可以通过 reflect.ValueOf(\&x).Elem()
来获得任意变量 x 可寻址的 Value 值

\begin{itemize}
\item
  可以通过变量的 CanAddr 方法来询问 reflect.Value 变量是否可寻址:
  \texttt{fmt.Println(x.CanAddr())}
\item
  从一个可寻址的 reflect. Value() 获取变量需要三步。首先,调用
  Addr(),返回一个Value,其中包含一个指向变量的指针,接下来,在这个
  Value 上调用 Interface(),会返回一个包含这个指针的 Interface\{\}
  值。最后,如果我们知道变量的类型,我们可以使用类型断言来把接口内容转换为一个普通指针。之后就可以通过这个指针来更新变量了
\end{itemize}

\begin{verbatim}
x := 2
d := reflect.ValueOf(&x).Elem()     // d 代表变量 x
p := d.Addr().Interface().(*int)    // px := &x
*px = 3                             // x = 3
fmt.Println(x)                      // "3"
\end{verbatim}

\begin{itemize}
\tightlist
\item
  还可以直接通过可寻址的 reflect.Value
  来更新变量,不用通过指针,而是直接调用reflect.Value.Set 方法:
  \texttt{d.Set(reflect.ValueOf(4))} 以及一些特化的 Set 变种 SetInt
  、SetUint 、SetString 、SetFloat 等: \texttt{d.SetInt(3)}
\end{itemize}

注: 1.
平常由编译器来检查的那些可赋值性条件,在这种情况下则是在运行时由Set
方法来检查 2. 变量和值类型需一致 3. 再不可寻址的 reflect.Value
上调用会崩溃 4.
反射可以读取到未导出的结构字段的值,但是不能更新它们,更新前可以用
CanSet 方法判断

\hypertarget{ux6ce8ux610fux4e8bux9879}{%
\subsection{12.9 注意事项}\label{ux6ce8ux610fux4e8bux9879}}

\hypertarget{ux614eux7528ux53cdux5c04ux7684ux4e09ux4e2aux539fux56e0}{%
\subsubsection{慎用反射的三个原因:}\label{ux614eux7528ux53cdux5c04ux7684ux4e09ux4e2aux539fux56e0}}

\begin{enumerate}
\def\labelenumi{\arabic{enumi}.}
\tightlist
\item
  基于反射的代码是很脆弱的。能导致编译器报告类型错误的每种写法,在反射中都有一个对应的误用方法。编译器在编译时就能向你报告这个错误,而反射错误则要等到执行时才以崩溃的方式来报告
\item
  类型其实也算是某种形式的文档,而反射的相关操作则无法做静态类型检查,所以大量使用反射的代码是很难理解的
\item
  基于反射的函数会比为特定类型优化的函数慢一两个数量级
\end{enumerate}
