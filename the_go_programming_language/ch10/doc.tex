\hypertarget{ux5305ux548cgoux5de5ux5177}{%
\subsection{包和go工具}\label{ux5305ux548cgoux5de5ux5177}}

\hypertarget{ux5305ux8defux5f84}{%
\paragraph{包路径}\label{ux5305ux8defux5f84}}

1.全局唯一性

2.应该以互联网域名作为开始

\begin{verbatim}
import ( 
"fmt"
"math/rand"
"encoding/json"
"golang.org/x/net/html" 
"github.com/go-sql-driver/mysql"
)
\end{verbatim}

\hypertarget{ux5305ux58f0ux660e}{%
\paragraph{包声明}\label{ux5305ux58f0ux660e}}

\begin{verbatim}
package xxx
\end{verbatim}

\hypertarget{ux5305ux5bfcux5165}{%
\paragraph{包导入}\label{ux5305ux5bfcux5165}}

\begin{verbatim}
import "fmt" 
import "os"

import ( 
    "fmt" 
    "os"
)
\end{verbatim}

重名导入

\begin{verbatim}
import ( 
    "crypto/rand"
    mrand "math/rand" // 通过指定一个不同的名称mrand就避免了冲突
    )
\end{verbatim}

每个导入声明从当前包向导入的包建立一个依赖 如果这些依赖形成一个循环,go
build 工具会报错

\hypertarget{ux7a7aux5bfcux5165}{%
\paragraph{空导入}\label{ux7a7aux5bfcux5165}}

\begin{verbatim}
import _ "image/png" //注册 PNG 解码器
\end{verbatim}

如果导人的包的名字没有在文件中引用,就会产生一个编译错误。但是,有时候,我们
必须导人一个包,这仅仅是为了利用其副作用:对包级别的变量执行初始化表达式求值,并
执行它的init 函数。

\hypertarget{goux5de5ux5177}{%
\paragraph{go工具}\label{goux5de5ux5177}}

go 工具(go
tool),它用来下载、查询、格式化、构建、测试以及安装Go代码包。

\hypertarget{ux5de5ux4f5cux7a7aux95f4ux7684ux7ec4ux7ec7}{%
\paragraph{工作空间的组织}\label{ux5de5ux4f5cux7a7aux95f4ux7684ux7ec4ux7ec7}}

在安装go工具时,通过GOPATH环境变量来指定工作空间的根。而在工作空间下有三个子目录,他们分别代表着:
src:源文件目录,每一个包放在一个目录中。
pkg:构建工具存储编译后的包的位置。 bin:放置可执行程序。

\hypertarget{ux5305ux7684ux4e0bux8f7d}{%
\paragraph{包的下载}\label{ux5305ux7684ux4e0bux8f7d}}

go get 命令 go get -u 可以指定下载最新的版本的包

\hypertarget{ux5305ux7684ux6784ux5efa}{%
\paragraph{包的构建}\label{ux5305ux7684ux6784ux5efa}}

go build 命令

如果包的名字是main,则go
build命令调用链接器在当前目录中创建可执行程序,可执行程序的名字取自包的导入路径的最后一段。

go run命令可以将包的构建和运行合并起来

\begin{verbatim}
$ go run quoteags.go one "two three" four\ five 
["one" "two three" "four five"]
\end{verbatim}

第一个不是以 go 文件结尾的参数会作为 Go 可执行程序的参数列表的开始

go install 令和 go
build非常相似,区别是它会保存每一个包的编译代码和命令,而不是把它们丢弃,而是保存在\$GOPATH/pkg目录中。这样于没有改变的包和命令不需要重新编译,从而使后续的构建更加快速。

\hypertarget{ux6784ux5efaux6807ux7b7e}{%
\paragraph{构建标签}\label{ux6784ux5efaux6807ux7b7e}}

在包的声明之前的注释

\begin{verbatim}
//+build linux darwin  表明go build只会在构建Linux 或者Mac OS X 系统应用的时候才会对它进行编译
//+build ignore  任何时候都不要编译这个文件:
\end{verbatim}

\hypertarget{ux5305ux7684ux6587ux6863ux5316}{%
\paragraph{包的文档化}\label{ux5305ux7684ux6587ux6863ux5316}}

\begin{verbatim}
// Fprintf 根据格式说明符格式化并写入w
// 返回写入的字节数及可能遇到的错误
func Fprintf(w io.Writer,format string,a ... interface{}) (int,error)
\end{verbatim}

包声明的前面的文档注释被认为是整个包的文档注释,且只有一个。如果包的注释比较长可以使用一个注释文件,通常叫做doc.go

go doc 工具输出在命令行上指定的内容的声明和整个文档注释.

godoc工具 它提供相互链接的HTM 页面服务,进而提供不少于go doc命令的信息

\begin{verbatim}
$ godoc -http :8000
\end{verbatim}

\hypertarget{ux5185ux90e8ux5305}{%
\paragraph{内部包}\label{ux5185ux90e8ux5305}}

这是包用来封装Go程序最重要的机制。没有导出的标识符只能在同一个包内访问,导出的标识符可以在世界任何地方访问。

go build
工具会特殊对待导人路径中包含路径片段internal的情况,这些包叫内部包。

\begin{verbatim}
net/http 
net/http/internal/chunked (内部包)
net/http/httputil 
net/url
\end{verbatim}

net/http/internal/chunked
(内部包)可以从net/http或者net/http/httputil导入,但是不能从net/url导入。然而,net/url
可以导入 net/http/httputil。

\hypertarget{ux5305ux7684ux67e5ux8be2}{%
\paragraph{包的查询}\label{ux5305ux7684ux67e5ux8be2}}

go list
工具上报可用包的信息。判断一个包是否存在于工作空间中,如果存在输出它的导人路径。

可使用通配符\texttt{...},用来匹配包的导入路径中的任意字串。

\begin{verbatim}
$ go list ... //枚举一个go工作空间的所有包
$ go list  gopl.io/ch3/... //列举指定子树中的所有包
$ go list ...xml... //一个具体的主题
\end{verbatim}

go list 命令获取每
个包的完整元数据,而不仅仅是导人路径,并且提供各种对于用户或者其他工具可访问的格式。

-json 标记使go list以JSON格式输出每个包的完整记录

\begin{verbatim}
$ go list -json hash
\end{verbatim}

-f 标记可以让用户通过text/template包提供的模板语言定制输出格式。
