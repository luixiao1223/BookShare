\documentclass[serif,mathserif]{beamer}
\usepackage{CJKutf8}
\usepackage{amsmath, amsfonts, epsfig, xspace}
\usepackage{algorithm,algorithmic}
\usepackage{pstricks,pst-node}
\usepackage{graphicx}
\usepackage{bm}
\usepackage{pdfrender}
\usepackage{tikz}
\usepackage{tcolorbox}
\usepackage{xcolor}
\usepackage{multimedia}
\usepackage[normal,tight,center]{subfigure}
\setlength{\subfigcapskip}{-.5em}
\usepackage{beamerthemesplit}

\usetheme{lankton-keynote}


\author[frame work]{https://github.com/taichi-dev/taichi}

\title[taichi\hspace{2em}\insertframenumber/\inserttotalframenumber]{Taichi}

% \date{February 14, 2008} %leave out for today's date to be insterted

\begin{document}
\begin{CJK}{UTF8}{gbsn}
\maketitle

% \section{Introduction}  % add these to see outline in slides

\begin{frame}
  \frametitle{Content}
  \begin{enumerate}
  \item \textcolor{red}{Installation}
  \item Initialization
  \item Data types
  \item Kernels
  \item Functions
  \item Parallel for-loops
  \item Range-for loops
  \item Struct-for loops
  \item Atomic Operations
  \item Taichi-scope v.s. Python-scope
  \end{enumerate}
\end{frame}

\begin{frame}[fragile]
  \frametitle{Installation}
  Taichi can be installed via pip on 64-bit Python 3.6,3.7,3.8
  \begin{tcolorbox}
\begin{verbatim}
 pip install taichi
\end{verbatim}
  \end{tcolorbox}

  \begin{itemize}
  \item Taichi supports Windows, Linux, and OS X
  \item Taichi runs on both CPUs and GPUs (CUDA,OpenGL,Apple Metal).
  \end{itemize}
\end{frame}

\begin{frame}
  \frametitle{Content}
  \begin{enumerate}
  \item Installation
  \item \textcolor{red}{Initialization}
  \item Data types
  \item Kernels
  \item Functions
  \item Parallel for-loops
  \item Range-for loops
  \item Struct-for loops
  \item Atomic Operations
  \item Taichi-scope v.s. Python-scope
  \end{enumerate}
\end{frame}

\begin{frame}[fragile]
  \frametitle{Initialization}

  \begin{tcolorbox}
\begin{verbatim}
ti.init(arch=ti.cuda)
\end{verbatim}
  \end{tcolorbox}

  \begin{itemize}
  \item ti.x64[arm,cuda,opengl,metal]: stick to a certain backend.
  \item ti.cpu (default)
  \item ti.gpu[cuda,metal,opengl]
  \end{itemize}

\end{frame}

\begin{frame}
  \frametitle{Content}
  \begin{enumerate}
  \item Installation
  \item Initialization
  \item \textcolor{red}{Data types}
  \item Kernels
  \item Functions
  \item Scalar math
  \item Matrices and linear algebra
  \item Parallel for-loops
  \item Range-for loops
  \item Struct-for loops
  \item Atomic Operations
  \item Taichi-scope v.s. Python-scope
  \end{enumerate}
\end{frame}


\begin{frame}[fragile]
  \frametitle{Data types}
  \begin{tcolorbox}
\begin{verbatim}
ti.i8/i16/i32/i64
ti.u8/u16/u32/u64
ti.f32/f64
\end{verbatim}
  \end{tcolorbox}
  \begin{itemize}
  \item \textcolor{green}{tensors}
    \begin{itemize}
    \item scalar: ti.field
    \item vector: ti.Vector
    \item matrix: ti.Matrix
    \end{itemize}
  \end{itemize}
  eg: ex01.py
\end{frame}

\begin{frame}
  \frametitle{Content}
  \begin{enumerate}
  \item Installation
  \item Initialization
  \item Data types
  \item \textcolor{red}{Kernels}
  \item Functions
  \item Parallel for-loops
  \item Range-for loops
  \item Struct-for loops
  \item Atomic Operations
  \item Taichi-scope v.s. Python-scope
  \end{enumerate}
\end{frame}

\begin{frame}
  \frametitle{Kernels}
  \textcolor{green}{Computation resides in kernels.}
  \begin{itemize}
  \item compiled, 
  \item statically-typed, 
  \item parallel 
  \item differentiable
  \end{itemize}
  eg: ex02.py
  
\end{frame}

\begin{frame}
  \frametitle{Content}
  \begin{enumerate}
  \item Installation
  \item Initialization
  \item Data types
  \item Kernels
  \item \textcolor{red}{Functions}
  \item Parallel for-loops
  \item Range-for loops
  \item Struct-for loops
  \item Atomic Operations
  \item Taichi-scope v.s. Python-scope
  \end{enumerate}
\end{frame}

\begin{frame}
  \frametitle{Functions}
  \begin{itemize}
  \item Taichi functions can be called by Taichi kernels and other Taichi functions
  \item Taichi functions will be force-inlined
  \end{itemize}

  eg: ex03.py

  
\end{frame}

\end{CJK}
\end{document}

