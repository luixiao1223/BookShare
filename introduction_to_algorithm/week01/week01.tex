% Created 2019-10-31 Thu 13:26
% Intended LaTeX compiler: pdflatex
\documentclass[11pt]{article}
\usepackage[utf8]{inputenc}
\usepackage[T1]{fontenc}
\usepackage{graphicx}
\usepackage{grffile}
\usepackage{longtable}
\usepackage{wrapfig}
\usepackage{rotating}
\usepackage[normalem]{ulem}
\usepackage{amsmath}
\usepackage{textcomp}
\usepackage{amssymb}
\usepackage{capt-of}
\usepackage{hyperref}

\usepackage{CJKutf8}
\usepackage{geometry}
\geometry{left=2.0cm,right=2.0cm,top=2.0cm,bottom=2.0cm}

\author{luixiao1223}
\date{\today}
\title{week01}
\hypersetup{
 pdfauthor={luixiao1223},
 pdftitle={week01},
 pdfkeywords={},
 pdfsubject={},
 pdfcreator={Emacs 26.3 (Org mode 9.2.6)}, 
 pdflang={English}}
\begin{document}
\begin{CJK}{UTF8}{gbsn}
  
\maketitle

\href{https://ocw.mit.edu/courses/electrical-engineering-and-computer-science/6-006-introduction-to-algorithms-fall-2011/lecture-videos/MIT6\_006F11\_lec01.pdf}{MIT-lecture01-http-link(click here)}

\section{Peak Finding}

\subsection{One-dimensional Version}

\subsubsection{Peak 的定义}
\begin{equation}
a[i-1]\leqslant a[i] \leqslant a[i+1] \Rightarrow \text{a[i] is a peak}
\end{equation}

如果在边界上?

\begin{equation}
a[0]>a[1] \Rightarrow \mbox{a[0] is a peak}
\end{equation}

\begin{equation}
a[n]>a[n-1]\Rightarrow \mbox{a[n] is a peak}
\end{equation}

\subsubsection{example}

\begin{center}
  \begin{tabular}{|l|r|r|r|r|r|}
    \hline
    index & a[0] & a[1] & a[2] & a[3] & a[4]\\
    \hline
    val & 4 & 5 & 4 & 3 & 6\\
    \hline
  \end{tabular}
\end{center}

peaks = \(\{5,6\}\)

\subsubsection{问题}

\begin{itemize}
\item \textbf{Find a peak if it exists.}
\end{itemize}

\begin{enumerate}
\item 这样的peak总是存在么?
  \begin{itemize}
  \item 利用反证法,假设数组a里面没有peak。
  \item 如果没有peak,那么$a[0]$不可以是peak,根据定义,$a[0]<a[1]$
  \item 同理因为a中没有peak,且$a[0]<a[1]$,为了使$a[1]$不是peak,$a[1]$必须小于$a[2]$。
  \item 以此类推$a[i]$必须小于$a[i+1]$.
  \item 最后$a[n-1]\leqslant a[n] \Rightarrow \mbox{a[n] is a peak}$与假设矛盾
  \item 结论,peak总是存在
  \end{itemize}
\item 如果把定义中的$\leqslant ,\geqslant $替换为$<,>$号。那么peak总是存在么?

  \begin{center}
    \begin{tabular}{|l|r|r|r|r|r|}
      \hline
      index & a[0] & a[1] & a[2] & a[3] & a[4]\\
      \hline
      val & 4 & 4 & 4 & 4 & 4\\
      \hline
    \end{tabular}
  \end{center}
\end{enumerate}

\subsubsection{最直接的方法}

\begin{verbatim}
if a[0] >= a[1] 
    return True, 0

for i = 1 to n-2
    if a[i-1] <= a[i] and a[i] >= a[i+1]
        return True, i

if a[n] >= a[n-1]
    return True,n
\end{verbatim}
\textbf{时间复杂度} $\Theta(n)$,有没有更快的方法?

\subsubsection{更快的方法-分治法}

\textbf{分治法的主要步骤}

\begin{itemize}
\item 把原问题划分为规模更小的子问题。子问题和原问题是同样的问题,只是规模更小。
\item 求解子问题。
\item 合并子问题。最终得出结果
\end{itemize}

\textbf{具体算法}

\begin{itemize}
\item if $a[n / 2]<a[n / 2-1]$ then only look at left half
\item if $a[n / 2]<a[n / 2+1]$ then only look at right half
\item else return $n/2$
\end{itemize}

\textbf{时间复杂度}:$\Theta(log_2(n))$,时间复杂度怎么计算出来的。

\begin{displaymath}
  T(n)=T(n/2)+\Theta(1)
\end{displaymath}

\begin{displaymath}
  \left.
    \begin{aligned}
      &T(n)-T(n/2)=\Theta(1)\\
      &T(n/2)-T(n/4)=\Theta(1)\\
      \vdots \\
      &T(4/2)-T(1) = \Theta(1)
    \end{aligned}
  \right.
\end{displaymath}

有多少个等式?$log_2(n)$个。

\begin{displaymath}
  T(n)-T(1)=log_2(n)\Theta(1)=\Theta(log_2(n)), T(1)=\Theta(1)
\end{displaymath}

\subsubsection{新的分治法快了多少?}

\begin{itemize}
\item compare $n$ and $log_2(n)$.
\item Transform
  \begin{displaymath}
    log_2(n)=b \Rightarrow n=2^b
  \end{displaymath}

  we map $log_2(n)\mapsto b, n\mapsto 2^b$

  compare $2^b$ and $b$
\end{itemize}

由此可见这是一个巨大的算法提升。

\subsection{Tow-dimensional Version}

\begin{center}
  \begin{tabular}{|c|c|c|c|}
    \hline
     & c & &  \\\hline
    b & a & d & \\\hline
    & e & & \\\hline
    & & & \\
    \hline
  \end{tabular}
\end{center}

a is a 2D-peak iff $a \geq b, a \geq d, a \geq c, a \geq e$

\subsubsection{straightforward method}

\begin{verbatim}
for i = 1 to m
    for j = 1 to n
        Is a[i][j] a peak?
\end{verbatim}

时间复杂度$\Theta(mn)$

\subsubsection{贪心算法}

\begin{center}
  \begin{tabular}{|c|c|c|c|}
    \hline
    3 & 3 & 5 & 4 \\\hline
    14 & 13 & 12 & 5 \\\hline
    15 & 9 & 11 & 17 \\\hline
    16 & 17 & 19 & 20 \\
    \hline
  \end{tabular}
\end{center}

\begin{itemize}
\item 随机从某个位置开始
\item 然后挑选它的临接点中大于它的元素作为下一个搜索点。如果没有,那么此节点就是peak,重复此步骤。
\end{itemize}

\textbf{利用贪心算法来证明必然有2-D peak}

\begin{itemize}
\item 贪心算法必然终止
\item 终止位置必然是peak
\end{itemize}

\textbf{贪心算法的复杂度}

最坏情况复杂度$\Theta(mn)$,if n = m, $\Theta(n^2)$。其中m,n为数组的两个维度。

\subsubsection{更快的算法}

如何来设计更快的算法,\textbf{分治法}

\begin{itemize}
\item 如何把原来的数组分开?
  
\end{itemize}

\subsubsection{扩展}

3-D peak,4-D peak,...,n-D peak?

\section{渐进符号}

渐进符号一共有5个($O,\Omega,\Theta, o, \omega$),其中3个($O,\Omega,\Theta$)最为重要也最为常用。

\begin{itemize}
\item $O(g(n))$
  \begin{displaymath}
    O(g(n))=\{f(n): f(n) \leqslant c g(n)\},c\mbox{ is a constant}
  \end{displaymath}

\item $\Theta(g(n))$

  \begin{displaymath}
    \Theta(g(n))=\{f(n):c_1g(n)\leqslant f(n) \leqslant c_2 g(n)\},c_1,c_2,\mbox{are constants}
  \end{displaymath}

  在分析算法复杂度的时候我们使用$f(n)=\Theta(g(n))$,其实是说$f(n)\in \Theta(g(n))$

\item $\Omega(g(n))$

  \begin{displaymath}
    \Omega(g(n))=\{f(n):cg(n)\leqslant f(n) \},c\mbox{ is a constant}
  \end{displaymath}
  
\item $o(g(n))$
  \begin{displaymath}
    O(g(n))=\{f(n): f(n) < c g(n)\},c\mbox{ is a constant}, \lim_{n\rightarrow \infty} \frac{g(n)}{f(n)}=\infty
  \end{displaymath}

  \item $\omega(g(n))$

  \begin{displaymath}
    \omega(g(n))=\{f(n):cg(n)< f(n) \},c\mbox{ is a constant} \lim_{n\rightarrow \infty} \frac{g(n))}{f(n)}=0
  \end{displaymath}
  
\end{itemize}

\subsection{类比记忆}


\begin{center}
  \begin{tabular}{|c|c|}
    \hline
    $f(n)=O(g(n))$ & $f(n)\leqslant g(n)$  \\\hline
    $f(n)=\Theta(g(n))$ & $f(n) = g(n)$  \\\hline
    $f(n)=\Omega(g(n))$ & $f(n) \geqslant  g(n)$  \\\hline
    $f(n)=o(g(n))$ & $f(n) < g(n) $ \\\hline
    $f(n)=\omega(g(n))$ & $f(n) > g(n)$ \\
    \hline
  \end{tabular}
\end{center}


\end{CJK}
\end{document}

