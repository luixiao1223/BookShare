% Created 2019-10-31 Thu 13:26
% Intended LaTeX compiler: pdflatex
\documentclass[11pt]{article}
\usepackage[utf8]{inputenc}
\usepackage[T1]{fontenc}
\usepackage{graphicx}
\usepackage{grffile}
\usepackage{longtable}
\usepackage{wrapfig}
\usepackage{rotating}
\usepackage[normalem]{ulem}
\usepackage{amsmath}
\usepackage{textcomp}
\usepackage{amssymb}
\usepackage{capt-of}
\usepackage{hyperref}

\usepackage{CJKutf8}
\usepackage{geometry}
\geometry{left=2.0cm,right=2.0cm,top=2.0cm,bottom=2.0cm}

\author{luixiao1223}
\date{\today}
\title{week01}
\hypersetup{
 pdfauthor={luixiao1223},
 pdftitle={week01},
 pdfkeywords={},
 pdfsubject={},
 pdfcreator={Emacs 26.3 (Org mode 9.2.6)}, 
 pdflang={English}}
\begin{document}
\begin{CJK}{UTF8}{gbsn}
  
\href{https://ocw.mit.edu/courses/electrical-engineering-and-computer-science/6-006-introduction-to-algorithms-fall-2011/lecture-videos/MIT6\_006F11\_lec01.pdf}{MIT-lecture01-http-link(click here)}

\section{Peak Finding}
\subsubsection{新的分治法快了多少?}
\begin{itemize}
\item compare $n$ and $log_2(n)$.
\item Transform
  \begin{displaymath}
    log_2(n)=b \Rightarrow n=2^b
  \end{displaymath}

  we map $log_2(n)\mapsto b, n\mapsto 2^b$,compare $2^b$ and $b$
\end{itemize}

由此可见这是一个巨大的算法提升。(参见mathematica图形)

\section{渐进符号}

渐进符号一共有5个($O,\Omega,\Theta, o, \omega$),其中3个($O,\Omega,\Theta$)最为重要也最为常用。

\begin{itemize}
\item $O(g(n))$
  \begin{displaymath}
    O(g(n))=\{f(n): f(n) \leqslant c g(n)\},c\mbox{ is a constant}
  \end{displaymath}

\item $\Theta(g(n))$

  \begin{displaymath}
    \Theta(g(n))=\{f(n):c_1g(n)\leqslant f(n) \leqslant c_2 g(n)\},c_1,c_2,\mbox{are constants}
  \end{displaymath}

  在分析算法复杂度的时候我们使用$f(n)=\Theta(g(n))$,其实是说$f(n)\in \Theta(g(n))$

\item $\Omega(g(n))$

  \begin{displaymath}
    \Omega(g(n))=\{f(n):cg(n)\leqslant f(n) \},c\mbox{ is a constant}
  \end{displaymath}
  
\item $o(g(n))$
  \begin{displaymath}
    O(g(n))=\{f(n): f(n) < c g(n)\},c\mbox{ is a constant}, \lim_{n\rightarrow \infty} \frac{g(n)}{f(n)}=\infty
  \end{displaymath}

  \item $\omega(g(n))$

  \begin{displaymath}
    \omega(g(n))=\{f(n):cg(n)< f(n) \},c\mbox{ is a constant} \lim_{n\rightarrow \infty} \frac{g(n))}{f(n)}=0
  \end{displaymath}
  
\end{itemize}

\subsection{类比记忆}


\begin{center}
  \begin{tabular}{|c|c|}
    \hline
    $f(n)=O(g(n))$ & $f(n)\leqslant g(n)$  \\\hline
    $f(n)=\Theta(g(n))$ & $f(n) = g(n)$  \\\hline
    $f(n)=\Omega(g(n))$ & $f(n) \geqslant  g(n)$  \\\hline
    $f(n)=o(g(n))$ & $f(n) < g(n) $ \\\hline
    $f(n)=\omega(g(n))$ & $f(n) > g(n)$ \\
    \hline
  \end{tabular}
\end{center}


\end{CJK}
\end{document}

